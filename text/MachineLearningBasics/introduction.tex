Over 70 years ago in October of 1950 computing hardware was still in its infancy. At a time when computers weighed several tons, could only perform a few thousand operations per second and the pinnacle of machine intelligence were analogous robots that could follow light sources \cite{FirstThinkingMachinesArticle}, Alan Turing published a paper in the journal of Nature discussing the question "Can machines think?" \cite{TuringThinkingPaper}. In this paper, Turing tries to tackle the question by proposing a game he calls the "imitation game". This game puts a human, whom we will refer to as Alice, in a room where she can communicate via written messages with two different entities, one of which being a human called Bob, the other being a machine. Alice's goal is to determine from this simple communication alone which of the two entities is the human. The goal of both the machine and Bob is to convince Alice that they are the human.\\
The largest execution of such an experiment to date took place in early 2023 in the form of an online chat portal where players had two minutes to talk to either another human or an Artificial Intelligence (AI) without knowing the type of their interlocutor \cite{TuringGamePaper}. After more than two million participants had played the game for a total of more than 15 million conversations, only \SI{68}{\percent} of the attempted classifications were correct.\\
All of the advanced AI-bots used in this experiment were achieved using machine learning methods. The Oxford Learning Dictionary defines machine learning as "a type of artificial intelligence in which computers use large amounts of data to learn how to do tasks rather than being programmed to do them" \cite{MLDefinition}. The theoretical foundation of such will be explained in the following sections.\\
\cref{sec:NeuralNetworks(BigSection)} discusses the workings of neurons and how they form neural networks. They consist of a fixed part, referred to as architecture, and modifiable parameters. These networks will be used as functions which replicate the relation of inputs and outputs in given datasets. \cref{sec:NeuralNetworkTraining} covers how these datasets have to be structured and how one can find the optimal parameters for which the network replicates the given datasets.
