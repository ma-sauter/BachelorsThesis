Over 70 years ago in October of 1950, when computers weighed several tons and could perform only a few thousand operations per second, Alan Turing published a paper in the journal of Nature titled "I.-COMPUTING MACHINERY AND INTELLIGENCE"\cite{TuringThinkingPaper}. At a time when the pinnacle of machine intelligence were analogous robots that could only shakily follow light sources\cite{FirstThinkingMachinesArticle}, he addressed the question "Can machines think?". He tries to tackle it by replacing by proposing a game he calls the "imitation game". This game puts a human, whom we will call A , in a room where he can communicate via written messages with two different entities, one of which being a machine, the other being a human called B. The goal of A is to identify through communicating with the machine and B, which of the two entities is the human being. The goal of both the machine and B is convincing A of themselves not being a machine.\\
The largest execution of such an experiment to date took place in early 2023 in the form of an online chat portal, where players had two minutes to talk to either another human or an artificial intelligence without knowing the type of their interlocutor. After more than two million participants had played the game for a total of more than 15 million conversations, only \SI{68}{\percent} of the attempted classifications were correct guesses. All of the advanced AI-bots used in this experiment were achieved using machine learning methods.\\
The Oxford Learning Dictionary defines machine learning as "a type of artificial intelligence in which computers use large amounts of data to learn how to do tasks rather than being programmed to do them"\cite{MLDefinition}. How we can do this with neural networks and use them to learn complex tasks is explained in the following sections.

