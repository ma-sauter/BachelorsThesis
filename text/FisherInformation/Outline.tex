Before starting this chapter, let's go over the ideas behind its structure.\\
First, \cref{sec:FIinStatistics} will describe the Fisher information as it is used in statistics. Some readers may wonder why we're discussing a statistical method describing probabilities when we've only talked about machine learning and neural networks before. The answer is, that the Fisher information matrix also
acts as the Riemannian metric describing the statistical manifold of the network regarding its loss, which is explained in \cref{sec:FisherInformationAsRiemannianMetric(BigChapter)}. To make the mathematically abstract concepts in this section more accessible, \cref{sec:FisherInterpretation} provides a brief recap with some added intuitive explanations. To conclude this chapter, \cref{sec:FIPhysics} presents another application of the Fisher information in a physical context, where it can be used to find phase transitions in thermodynamic systems.