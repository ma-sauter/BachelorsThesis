\usepackage[T1]{fontenc}
\usepackage[ngerman, english]{babel}
\usepackage{siunitx}
\usepackage{graphicx}
\usepackage{tipa} % for the \ark{} command
\usepackage{graphics} % for pdf, bitmapped graphics files
\usepackage{times} % assumes new font selection scheme installed
\usepackage{amsmath}
\usepackage{latexsym}
\usepackage{amscd}% for commutative diagrams
\usepackage{mathrsfs} %this package is for the script font \mathscr
\usepackage{relsize}
\usepackage{delarray}
\usepackage{pstricks}
\usepackage{theorem}
\usepackage{changepage}
\usepackage{euscript}
\usepackage{textcomp}
\usepackage{esvect}
\usepackage{parskip}
%\usepackage{placeins}
\usepackage{subfigure}
% \usepackage{subcaption}
\usepackage{array}
\usepackage{delarray}
\usepackage{stmaryrd}
\usepackage{fancyhdr}
\usepackage{graphpap}
\usepackage{makeidx}
\usepackage{enumerate}
\usepackage{esint}
\usepackage{datetime}
\usepackage{caption}
\usepackage{smartdiagram}
\usesmartdiagramlibrary{additions}
%Set Abstract Page
\usepackage{abstract}
\setlength{\absleftindent}{-5mm}
\setlength{\absrightindent}{-5mm}

%Colour definitions - put before TikZ
%\usepackage{color}
%\definecolor{igreen}{rgb}{0.0, 0.56, 0.0}
%\usepackage{colortbl}
%\colorlet{gred}{-red!75!green!65!}
%\colorlet{mamber}{-red!75!green!15!blue!50!}
%\colorlet{grown}{-red!75!blue!20!green}
%\colorlet{bled}{-red!85!blue!40!green!45!}
%\colorlet{waters}{cyan!25} % Define color for the water
%\colorlet{water}{cyan!25!green!20!} % Define color for the water
%\definecolor{grin}{HTML}{00F9DE}
\usepackage{rotating}
\providecommand{\keywords}[1]{\textbf{\textit{Keywords---}} #1}

% For faint dotted table line
\usepackage{arydshln}
\setlength{\dashlinedash}{.4pt}
\setlength{\dashlinegap}{.8pt}

\usepackage{booktabs}
\usepackage{graphicx}
\usepackage{tikz}
\usepackage{tikz-3dplot}
\usetikzlibrary{
	arrows,
	arrows.meta,
	automata,
	backgrounds,
	calc,
	decorations,
	decorations.pathmorphing,
	decorations.pathreplacing,
	decorations.fractals,
	external,
	fit,
	matrix,
	petri,
	positioning,
	shadows,
	shapes,
	shapes.multipart,
	topaths,
	intersections
}
\usepackage{eso-pic}
\def\ba{\begin{array}}
	\def\ea{\end{array}}
\def\beann{\begin{eqnarray*}}
	\def\eeann{\end{eqnarray*}}
\def\bea{\begin{eqnarray}}
	\def\eea{\end{eqnarray}}
\def\bsy{\boldsymbol}
\def\gray#1{{\color{gray}#1}}

%% COUNTERS
\setcounter{MaxMatrixCols}{20}
\renewcommand{\thesection}{\arabic{section}}
\renewcommand{\thesection}{\thechapter.\number\numexpr\value{section}}
\renewcommand{\thesubsection}{\thesection.\number\numexpr\value{subsection}}
%%For changemargin
\def\quote{\list{}{\rightmargin\leftmargin}\item[]}
\let\endquote=\endlist 
\def\changemargin#1#2{\list{}{\rightmargin#2\leftmargin#1}\item[]}
\let\endchangemargin=\endlist 
\makeatletter
\newlength\qvec@height
\newlength\qvec@depth
\newlength\qvec@width
\newcommand{\qvec}[2][]{
	\settoheight{\qvec@height}{$#2$}
	\settodepth{\qvec@depth}{$#2$}
	\settowidth{\qvec@width}{$#2$}
	\def\qvec@arg{#1}
	\raisebox{.2ex}{\raisebox{\qvec@height}{\rlap{% 
				\kern.05em
				\begin{tikzpicture}[scale=1,shorten >=-3pt,shorten <=-3pt]
					\pgfsetroundcap
					\coordinate (Stx) at (.05em,0) ;
					\coordinate (Arx) at (\qvec@width-.05em,0) ;
					\draw[->](Stx) to[bend left] (Arx);
				\end{tikzpicture}
	}}}
	#2
}
\makeatother
\makeatletter
\newlength\pvec@height
\newlength\pvec@depth
\newlength\pvec@width
\newcommand{\pvec}[2][]{
	\settoheight{\pvec@height}{$#2$}
	\settodepth{\pvec@depth}{$#2$}
	\settowidth{\pvec@width}{$#2$}
	\def\pvec@arg{#1}
	\raisebox{.2ex}{\raisebox{\pvec@height}{\rlap{% 
				\kern.05em
				\begin{tikzpicture}[scale=1,shorten >=-3pt,shorten <=-3pt]
					\pgfsetroundcap
					\coordinate (Stx) at (.05em,0) ;
					\coordinate (Arx) at (\pvec@width-.05em,0) ;
					\draw[->](Stx) to[bend right] (Arx);
				\end{tikzpicture}
	}}}
	#2
}
\makeatother
\makeatletter
\newlength\vvec@height%
\newlength\vvec@depth%
\newlength\vvec@width%
\newcommand{\vvec}[2][]{%
	\ifmmode%
	\settoheight{\vvec@height}{$#2$}%
	\settodepth{\vvec@depth}{$#2$}%
	\settowidth{\vvec@width}{$#2$}%
	\else 
	\settoheight{\vvec@height}{#2}%
	\settodepth{\vvec@depth}{#2}%
	\settowidth{\vvec@width}{#2}%
	\fi%
	\def\vvec@arg{#1}%
	\def\vvec@dd{:}%
	\def\vvec@d{.}%
	\raisebox{.2ex}{\raisebox{\vvec@height}{\rlap{%
				\kern.05em%
				\begin{tikzpicture}[scale=1]
					\pgfsetroundcap
					\draw (.05em,0)--(\vvec@width-.05em,0);
					\draw (\vvec@width-.05em,0)--(\vvec@width-.15em, .075em);
					\draw (\vvec@width-.05em,0)--(\vvec@width-.15em,-.075em);
					\ifx\vvec@arg\vvec@d%
					\fill(\vvec@width*.45,.5ex) circle (.5pt);%
					\else\ifx\vvec@arg\vvec@dd%
					\fill(\vvec@width*.30,.5ex) circle (.5pt);%
					\fill(\vvec@width*.65,.5ex) circle (.5pt);%
					\fi\fi%
				\end{tikzpicture}%
	}}}%
	#2%
}
\makeatother
\def\ba{\begin{array}}
	\def\ea{\end{array}}
\def\beann{\begin{eqnarray*}}
	\def\eeann{\end{eqnarray*}}
\def\bea{\begin{eqnarray}}
	\def\eea{\end{eqnarray}}
\def\bsy{\boldsymbol}
\def\gray#1{{\color{gray}#1}}
\usepackage{titlesec}
\usepackage{multirow}
%To reference within text
%\usepackage{hyperref}
\usepackage{lipsum}
\usepackage{tikz-cd}
\usepackage{float}
\usepackage{titling}
\usepackage{epigraph}
\usepackage[title, titletoc]{appendix}
\setlength\epigraphwidth{8cm}
\setlength\epigraphrule{0pt}

\titleformat{\chapter}{\normalfont\LARGE}{\thechapter\,$\vert$}{20pt}{\LARGE}{\setcounter{chapter}{0}}
\setlength{\headheight}{15pt}
\titlespacing*{\chapter}{0pt}{-70pt}{40pt} %Move chapter titles up
% Title page logos:
\makeatletter
\newcommand\BackgroundPicturea[4]{
	\setlength{\unitlength}{1pt}
	\put(0,\strip@pt\paperheight){
		\parbox[t]{\paperwidth}{
			\vspace{#2}\hspace{#3}
			\mbox{\includegraphics[scale=#4]{#1}}
}}}
\newcommand\BackgroundPictureb[4]{
	\setlength{\unitlength}{1pt}
	\put(0,\strip@pt\paperheight){
		\parbox[t]{\paperwidth}{
			\vspace{#2}\hspace{#3}
			\mbox{\includegraphics[scale=#4]{#1}}
}}}
\makeatother
% For my abbreviations
\newcommand{\abbrlabel}[1]{\makebox[3cm][l]{\textbf{#1}\ \dotfill}}
\newenvironment{abbreviations}{\begin{list}{}{\renewcommand{\makelabel}{\abbrlabel}}}{\end{list}}
% Line Spacing
\usepackage{setspace}
\setstretch{1.5}
%Set of command is for the changemargin environment
\def\quote{\list{}{\rightmargin\leftmargin}\item[]}
\let\endquote=\endlist 
\def\changemargin#1#2{\list{}{\rightmargin#2\leftmargin#1}\item[]}
\let\endchangemargin=\endlist
%Replace Contents to Table of Contents	
\addto\captionsenglish{
	\renewcommand{\contentsname}%
	{Table of Contents}
	\setcounter{tocdepth}{3}% Include \subsubsection in ToC
	\setcounter{secnumdepth}{3}% Number \subsubsection in ToC
}
\renewcommand{\listfigurename}{List of Figures}
\renewcommand{\listtablename}{List of Tables}


%%%%%%%%%%%%%%%%%%%%%%%%%%%%%
%things after here are my own templates, there may be a lot of doubles.

\usepackage{amsmath, amssymb,amsfonts} %AMS math packets
\usepackage{graphicx} %For pictures

\usepackage{datetime} %For dates

\usepackage[backend=biber, citestyle=nature, sorting=none]{biblatex} %For citation
\addbibresource{literature.bib} %imports the bibliography file

\usepackage{xcolor} %For coloring text and equations
\definecolor{magmadark}{RGB}{84,1,172}
\definecolor{magmabright}{RGB}{212,83,62}

\usepackage[bookmarks,colorlinks=true]{hyperref} %For referencing
\hypersetup{
	colorlinks,
	linktocpage,
	citecolor=black,
	filecolor=black,
	linkcolor=black,
	urlcolor=black
}
\numberwithin{equation}{section} % Section prefix for equation names

\usepackage{siunitx} %For units

\usepackage[section]{placeins} %Floats nicht über section Grenze, \FloatBarrier baut grenze 

\usepackage{pgfplots}
%\DeclareUnicodeCharacter{2212}{-}
\usepgfplotslibrary{groupplots, dateplot}
\usetikzlibrary{external, patterns, shapes.arrows, decorations.pathreplacing,calligraphy}
\pgfplotsset{compat=newest}
\usepackage[figure]{hypcap} %Links auf abbildungen springen auf das bild statt auf die caption - muss nach hyperref eingebunden werden, \capstart definiert sprungpunkt, bei [figure] kommt automatisch einer


%Für doppelten unterstrich:
\newcommand{\matrixvariable}[1]{\underline{\underline{\boldsymbol{\mathit{#1}}}}} 

%Für bilder \img{dateiname}{breite}{caption}{label}
\newcommand{\img}[4]{
	\begin{figure}
		\centering
		\includegraphics[width = #2]{#1}
		\caption{#3}
		\label{#4}
	\end{figure}
}
%Für tex.bilder \inp{dateiname}{caption}{label}
\newcommand{\inp}[3]{
	\begin{figure}
		\centering
		\input{#1}
		\caption{#2}
		\label{#3}
	\end{figure}
}
%für totale Ableitung
\newcommand{\tAbl}[2]{\frac{\mathrm{d}#1}{\mathrm{d}#2}}
%für partielle Ableitung
\newcommand{\pAbl}[2]{\frac{\partial#1}{\partial#2}}
%für Jacobians
\newcommand{\jacobian}[2]{\mathcal{J}_{\left[#1/#2\right]}}

\usepackage{mathtools} % for =: sign
\mathtoolsset{centercolon}
%\mathtoolsset{showonlyrefs} % only label equations that get mentioned

\usepackage{tikz-network} %for picturing neural networks

\usepackage[capitalise]{cleveref} %both for referencing

\usepackage{pdfpages} %For including titlepage

%\usepackage[lmargin=142pt,rmargin=95pt,tmargin=127pt,bmargin=123pt]{geometry}