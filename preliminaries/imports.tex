\usepackage[utf8]{inputenc} %utf8 codierung

\usepackage{amsmath, amssymb,amsfonts} %AMS math packets
\usepackage{graphicx} %For pictures

\usepackage{datetime} %For dates

\usepackage[usenames,dvipsnames]{xcolor} %For coloring text and equations

\usepackage[bookmarks,colorlinks=true]{hyperref} %For referencing
\hypersetup{
	colorlinks,
	linktocpage,
	citecolor=black,
	filecolor=black,
	linkcolor=black,
	urlcolor=black
}
\numberwithin{equation}{section} % Section prefix for equation names

\usepackage{siunitx} %For units

\usepackage[section]{placeins} %Floats nicht über section Grenze, \FloatBarrier baut grenze 

\usepackage{pgfplots}
%\DeclareUnicodeCharacter{2212}{-}
\usepgfplotslibrary{groupplots, dateplot}
\usetikzlibrary{external, patterns, shapes.arrows, decorations.pathreplacing,calligraphy}
\pgfplotsset{compat=newest}
\usepackage[figure]{hypcap} %Links auf abbildungen springen auf das bild statt auf die caption - muss nach hyperref eingebunden werden, \capstart definiert sprungpunkt, bei [figure] kommt automatisch einer


%Für doppelten unterstrich:
\newcommand{\matrixvariable}[1]{\underline{\underline{\boldsymbol{\mathit{#1}}}}} 

%Für bilder \img{dateiname}{breite}{caption}{label}
\newcommand{\img}[4]{
	\begin{figure}
		\centering
		\includegraphics[width = #2]{#1}
		\caption{#3}
		\label{#4}
	\end{figure}
}
%Für tex.bilder \inp{dateiname}{caption}{label}
\newcommand{\inp}[3]{
	\begin{figure}
		\centering
		\input{#1}
		\caption{#2}
		\label{#3}
	\end{figure}
}
%für totale Ableitung
\newcommand{\tAbl}[2]{\frac{\mathrm{d}#1}{\mathrm{d}#2}}
%für partielle Ableitung
\newcommand{\pAbl}[2]{\frac{\partial#1}{\partial#2}}

\usepackage{tikz-network} %for picturing neural networks

\usepackage{hyperref}
\usepackage{cleveref} %both for referencing