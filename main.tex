% Template Author: Clara Eleonore Pavillet
% Version: 1.0
% This work is licensed under a Creative Commons Attribution 4.0 International License.

\documentclass[12pt, a4paper]{report}
\usepackage[T1]{fontenc}
\usepackage[english]{babel}
\usepackage{siunitx}
\usepackage{graphicx}
\usepackage{tipa} % for the \ark{} command
\usepackage{graphics} % for pdf, bitmapped graphics files
\usepackage{times} % assumes new font selection scheme installed
\usepackage{amsmath}
\usepackage{latexsym}
\usepackage{amscd}% for commutative diagrams
\usepackage{mathrsfs} %this package is for the script font \mathscr
\usepackage{relsize}
\usepackage{delarray}
\usepackage{pstricks}
\usepackage{theorem}
\usepackage{changepage}
\usepackage{euscript}
\usepackage{textcomp}
\usepackage{esvect}
\usepackage{parskip}
%\usepackage{placeins}
\usepackage{subfigure}
% \usepackage{subcaption}
\usepackage{array}
\usepackage{delarray}
\usepackage{stmaryrd}
\usepackage{fancyhdr}
\usepackage{graphpap}
\usepackage{makeidx}
\usepackage{enumerate}
\usepackage{esint}
\usepackage{datetime}
\usepackage{caption}
\usepackage{smartdiagram}
\usesmartdiagramlibrary{additions}
%Set Abstract Page
\usepackage{abstract}
\setlength{\absleftindent}{-5mm}
\setlength{\absrightindent}{-5mm}

%Colour definitions - put before TikZ
%\usepackage{color}
%\definecolor{igreen}{rgb}{0.0, 0.56, 0.0}
%\usepackage{colortbl}
%\colorlet{gred}{-red!75!green!65!}
%\colorlet{mamber}{-red!75!green!15!blue!50!}
%\colorlet{grown}{-red!75!blue!20!green}
%\colorlet{bled}{-red!85!blue!40!green!45!}
%\colorlet{waters}{cyan!25} % Define color for the water
%\colorlet{water}{cyan!25!green!20!} % Define color for the water
%\definecolor{grin}{HTML}{00F9DE}
\usepackage{rotating}
\providecommand{\keywords}[1]{\textbf{\textit{Keywords---}} #1}

% For faint dotted table line
\usepackage{arydshln}
\setlength{\dashlinedash}{.4pt}
\setlength{\dashlinegap}{.8pt}

\usepackage{booktabs}
\usepackage{graphicx}
\usepackage{tikz}
\usepackage{tikz-3dplot}
\usetikzlibrary{
	arrows,
	arrows.meta,
	automata,
	backgrounds,
	calc,
	decorations,
	decorations.pathmorphing,
	decorations.pathreplacing,
	decorations.fractals,
	external,
	fit,
	matrix,
	petri,
	positioning,
	shadows,
	shapes,
	shapes.multipart,
	topaths,
	intersections
}
\usepackage{eso-pic}
\def\ba{\begin{array}}
	\def\ea{\end{array}}
\def\beann{\begin{eqnarray*}}
	\def\eeann{\end{eqnarray*}}
\def\bea{\begin{eqnarray}}
	\def\eea{\end{eqnarray}}
\def\bsy{\boldsymbol}
\def\gray#1{{\color{gray}#1}}

%% COUNTERS
\setcounter{MaxMatrixCols}{20}
\renewcommand{\thesection}{\arabic{section}}
\renewcommand{\thesection}{\thechapter.\number\numexpr\value{section}}
\renewcommand{\thesubsection}{\thesection.\number\numexpr\value{subsection}}
%%For changemargin
\def\quote{\list{}{\rightmargin\leftmargin}\item[]}
\let\endquote=\endlist 
\def\changemargin#1#2{\list{}{\rightmargin#2\leftmargin#1}\item[]}
\let\endchangemargin=\endlist 
\makeatletter
\newlength\qvec@height
\newlength\qvec@depth
\newlength\qvec@width
\newcommand{\qvec}[2][]{
	\settoheight{\qvec@height}{$#2$}
	\settodepth{\qvec@depth}{$#2$}
	\settowidth{\qvec@width}{$#2$}
	\def\qvec@arg{#1}
	\raisebox{.2ex}{\raisebox{\qvec@height}{\rlap{% 
				\kern.05em
				\begin{tikzpicture}[scale=1,shorten >=-3pt,shorten <=-3pt]
					\pgfsetroundcap
					\coordinate (Stx) at (.05em,0) ;
					\coordinate (Arx) at (\qvec@width-.05em,0) ;
					\draw[->](Stx) to[bend left] (Arx);
				\end{tikzpicture}
	}}}
	#2
}
\makeatother
\makeatletter
\newlength\pvec@height
\newlength\pvec@depth
\newlength\pvec@width
\newcommand{\pvec}[2][]{
	\settoheight{\pvec@height}{$#2$}
	\settodepth{\pvec@depth}{$#2$}
	\settowidth{\pvec@width}{$#2$}
	\def\pvec@arg{#1}
	\raisebox{.2ex}{\raisebox{\pvec@height}{\rlap{% 
				\kern.05em
				\begin{tikzpicture}[scale=1,shorten >=-3pt,shorten <=-3pt]
					\pgfsetroundcap
					\coordinate (Stx) at (.05em,0) ;
					\coordinate (Arx) at (\pvec@width-.05em,0) ;
					\draw[->](Stx) to[bend right] (Arx);
				\end{tikzpicture}
	}}}
	#2
}
\makeatother
\makeatletter
\newlength\vvec@height%
\newlength\vvec@depth%
\newlength\vvec@width%
\newcommand{\vvec}[2][]{%
	\ifmmode%
	\settoheight{\vvec@height}{$#2$}%
	\settodepth{\vvec@depth}{$#2$}%
	\settowidth{\vvec@width}{$#2$}%
	\else 
	\settoheight{\vvec@height}{#2}%
	\settodepth{\vvec@depth}{#2}%
	\settowidth{\vvec@width}{#2}%
	\fi%
	\def\vvec@arg{#1}%
	\def\vvec@dd{:}%
	\def\vvec@d{.}%
	\raisebox{.2ex}{\raisebox{\vvec@height}{\rlap{%
				\kern.05em%
				\begin{tikzpicture}[scale=1]
					\pgfsetroundcap
					\draw (.05em,0)--(\vvec@width-.05em,0);
					\draw (\vvec@width-.05em,0)--(\vvec@width-.15em, .075em);
					\draw (\vvec@width-.05em,0)--(\vvec@width-.15em,-.075em);
					\ifx\vvec@arg\vvec@d%
					\fill(\vvec@width*.45,.5ex) circle (.5pt);%
					\else\ifx\vvec@arg\vvec@dd%
					\fill(\vvec@width*.30,.5ex) circle (.5pt);%
					\fill(\vvec@width*.65,.5ex) circle (.5pt);%
					\fi\fi%
				\end{tikzpicture}%
	}}}%
	#2%
}
\makeatother
\def\ba{\begin{array}}
	\def\ea{\end{array}}
\def\beann{\begin{eqnarray*}}
	\def\eeann{\end{eqnarray*}}
\def\bea{\begin{eqnarray}}
	\def\eea{\end{eqnarray}}
\def\bsy{\boldsymbol}
\def\gray#1{{\color{gray}#1}}
\usepackage{titlesec}
\usepackage{multirow}
%To reference within text
%\usepackage{hyperref}
\usepackage{lipsum}
\usepackage{tikz-cd}
\usepackage{float}
\usepackage{titling}
\usepackage{epigraph}
\usepackage[title, titletoc]{appendix}
\setlength\epigraphwidth{8cm}
\setlength\epigraphrule{0pt}

\titleformat{\chapter}{\normalfont\LARGE}{\thechapter\,$\vert$}{20pt}{\LARGE}{\setcounter{chapter}{0}}
\setlength{\headheight}{15pt}
\titlespacing*{\chapter}{0pt}{-70pt}{40pt} %Move chapter titles up
% Title page logos:
\makeatletter
\newcommand\BackgroundPicturea[4]{
	\setlength{\unitlength}{1pt}
	\put(0,\strip@pt\paperheight){
		\parbox[t]{\paperwidth}{
			\vspace{#2}\hspace{#3}
			\mbox{\includegraphics[scale=#4]{#1}}
}}}
\newcommand\BackgroundPictureb[4]{
	\setlength{\unitlength}{1pt}
	\put(0,\strip@pt\paperheight){
		\parbox[t]{\paperwidth}{
			\vspace{#2}\hspace{#3}
			\mbox{\includegraphics[scale=#4]{#1}}
}}}
\makeatother
% For my abbreviations
\newcommand{\abbrlabel}[1]{\makebox[3cm][l]{\textbf{#1}\ \dotfill}}
\newenvironment{abbreviations}{\begin{list}{}{\renewcommand{\makelabel}{\abbrlabel}}}{\end{list}}
% Line Spacing
\usepackage{setspace}
\setstretch{1.5}
%Set of command is for the changemargin environment
\def\quote{\list{}{\rightmargin\leftmargin}\item[]}
\let\endquote=\endlist 
\def\changemargin#1#2{\list{}{\rightmargin#2\leftmargin#1}\item[]}
\let\endchangemargin=\endlist
%Replace Contents to Table of Contents	
\addto\captionsenglish{
	\renewcommand{\contentsname}%
	{Table of Contents}
	\setcounter{tocdepth}{3}% Include \subsubsection in ToC
	\setcounter{secnumdepth}{3}% Number \subsubsection in ToC
}
\renewcommand{\listfigurename}{List of Figures}
\renewcommand{\listtablename}{List of Tables}


%%%%%%%%%%%%%%%%%%%%%%%%%%%%%
%things after here are my own templates, there may be a lot of doubles.

\usepackage{amsmath, amssymb,amsfonts} %AMS math packets
\usepackage{graphicx} %For pictures

\usepackage{datetime} %For dates

\usepackage[backend=biber, sorting=none, citestyle=nature]{biblatex} %For citation
\addbibresource{literature.bib} %imports the bibliography file

\usepackage{xcolor} %For coloring text and equations

\usepackage[bookmarks,colorlinks=true]{hyperref} %For referencing
\hypersetup{
	colorlinks,
	linktocpage,
	citecolor=black,
	filecolor=black,
	linkcolor=black,
	urlcolor=black
}
\numberwithin{equation}{section} % Section prefix for equation names

\usepackage{siunitx} %For units

\usepackage[section]{placeins} %Floats nicht über section Grenze, \FloatBarrier baut grenze 

\usepackage{pgfplots}
%\DeclareUnicodeCharacter{2212}{-}
\usepgfplotslibrary{groupplots, dateplot}
\usetikzlibrary{external, patterns, shapes.arrows, decorations.pathreplacing,calligraphy}
\pgfplotsset{compat=newest}
\usepackage[figure]{hypcap} %Links auf abbildungen springen auf das bild statt auf die caption - muss nach hyperref eingebunden werden, \capstart definiert sprungpunkt, bei [figure] kommt automatisch einer


%Für doppelten unterstrich:
\newcommand{\matrixvariable}[1]{\underline{\underline{\boldsymbol{\mathit{#1}}}}} 

%Für bilder \img{dateiname}{breite}{caption}{label}
\newcommand{\img}[4]{
	\begin{figure}
		\centering
		\includegraphics[width = #2]{#1}
		\caption{#3}
		\label{#4}
	\end{figure}
}
%Für tex.bilder \inp{dateiname}{caption}{label}
\newcommand{\inp}[3]{
	\begin{figure}
		\centering
		\input{#1}
		\caption{#2}
		\label{#3}
	\end{figure}
}
%für totale Ableitung
\newcommand{\tAbl}[2]{\frac{\mathrm{d}#1}{\mathrm{d}#2}}
%für partielle Ableitung
\newcommand{\pAbl}[2]{\frac{\partial#1}{\partial#2}}
%für Jacobians
\newcommand{\jacobian}[2]{\mathcal{J}_{\left[#1/#2\right]}}

\usepackage{mathtools} % for =: sign
\mathtoolsset{centercolon}
%\mathtoolsset{showonlyrefs} % only label equations that get mentioned

\usepackage{tikz-network} %for picturing neural networks

\usepackage[capitalise]{cleveref} %both for referencing

\usepackage{pdfpages} %For including titlepage

%\usepackage[lmargin=142pt,rmargin=95pt,tmargin=127pt,bmargin=123pt]{geometry}
\hypersetup{pdftitle = Thesis, pdfauthor = {Marc Sauter}, pdfstartview=FitH, pdfkeywords = essay, pdfpagemode=FullScreen, colorlinks, anchorcolor = red, citecolor = blue, urlcolor=blue, filecolor=green, linkcolor=red, plainpages=false}
%%%%%%%%%%%%%%%%%%%%%%%%%%%%%%%%%%%%%%%%%%%%%%%%%%%%%%%%%%%%%%%%%%%%%%%
\pagestyle{fancy}
\rhead{ICP}
\chead{}
\lhead{University of Stuttgart}
\lfoot{\date{}}
\cfoot{}
\rfoot{\thepage}
% Top and Bottom Line Rules
\renewcommand{\headrulewidth}{0.4pt} %0.4pt
\renewcommand{\footrulewidth}{0.4pt}
\fancyheadoffset{9pt}
\fancyfootoffset{9pt}
% Line spacing
\renewcommand{\baselinestretch}{1.5} %1.5
%%%%%%%%%%%%%%%%%%%%%%%%%%%%%%%%%%%%%%%%%%%%%%%%%%%%%%%%%%%%%%%%%%%%%%%
\date{}

\title{Investigating the Evolution of Eisher Information for Neural Network Dynamics}
\author{\\ \Large{Marc Sauter}
	\\ ICP
	\\
	\\
	\\
	\\ University of stuttgart
	\\
	A thesis presented for the degree of \\ \textit{B.Sc.}
	\\ \\
	2023
}
%%%%%%%%%%%%%%%%%%%%%%%%%%%%%%%%%%%%%%%%%%%%%%%%%%%%%%%%%%%%%%%%%%%%%%%
\begin{document}
	% Adjust logo positions here
	\AddToShipoutPicture*{\BackgroundPicturea{Logos/UniStuttgartLogo.png}{0.7in}{5.0in}{1}} %The last value changes the size factor
	\AddToShipoutPicture*{\BackgroundPictureb{Logos/ICPLogo.png}{6.1in}{3in}{0.9}}
	\thispagestyle{headings}
	\maketitle
	\FloatBarrier
	\pagenumbering{roman}
	
	\newpage
	\thispagestyle{empty}
	\vspace*{\fill}
	\begin{center}
		Copyright \copyright  \thinspace 2023 by Marc Sauter \\ All Rights Reserved
	\end{center}
	\vspace*{\fill}
	\newpage
	\thispagestyle{empty}
	\epigraph{Algebra is like sheet music. The important thing isn't can you read music, its can you hear it.}{--- \textup{Cristopher Nolan}}
	
	\thispagestyle{empty}
	\chapter*{Acknowledgements}
	I would like to express ...
	
	
	\thispagestyle{empty}
	\chapter*{Declaration}
	I, Marc Sauter, declare that ...
	
	\vspace{3cm}
	\noindent\begin{tabular}{ll}
		\makebox[2.5in]{\hrulefill} & \makebox[2.5in]{\hrulefill}\\
		\textit{Signature} & \textit{Date}\\
	\end{tabular}
	
	\thispagestyle{empty}
	\begin{abstract}
		\lipsum[1-2]
		
		\keywords{Keyword1 - Keyword2 - Keyword3}
		% \vspace{-10mm} %To remove added white space after
	\end{abstract}
	\tableofcontents
	\thispagestyle{plain}
	\listoffigures
	\listoftables
	
	\chapter*{List of Abbreviations}
	\begin{abbreviations}
		\item[AI] Artificial Intelligence
		\item[ML] Machine Learning
		\item[ReLU] Rectified Linear Unit
		\item[NN] Neural Network
		\item[GD] Gradient Descent
		\item[SGD] Stochastic Gradient Descent
		\item[NTK] Neural Tangent Kernel
		\item[FI] Fisher Information
	\end{abbreviations}	
	
	\chapter{Machine Learning Basics}
	\pagenumbering{arabic}
	\section{Introduction to machine learning}\label{sec:MachineLearningIntroduction}
	Over 70 years ago in October of 1950, at a time when computers weighed several tons, could only perform a few thousand operations per second and the pinnacle of machine intelligence were analogous robots that could follow light sources \cite{FirstThinkingMachinesArticle}, Alan Turing published a paper in the journal of Nature discussing the question "Can machines think?" \cite{TuringThinkingPaper}. In this paper, Turing tries to tackle the question by proposing a game he calls the "imitation game". This game puts a human, whom we will refer to as Alice, in a room where she can communicate via written messages with two different entities, one of which being a human called Bob, the other being a machine. Alice's goal is to determine from this simple communication alone which of the two entities is the human. The goal of both the machine and Bob is to convince Alice that they are the human.\\
The largest execution of such an experiment to date took place in early 2023 in the form of an online chat portal where players had two minutes to talk to either another human or an Artificial Intelligence (AI) without knowing the type of their interlocutor. After more than two million participants had played the game for a total of more than 15 million conversations, only \SI{68}{\percent} of the attempted classifications were correct guesses.\\
All of the advanced AI-bots used in this experiment were achieved using machine learning methods. The Oxford Learning Dictionary defines machine learning as "type of artificial intelligence in which computers use large amounts of data to learn how to do tasks rather than being programmed to do them" \cite{MLDefinition}. The theoretical foundation of such will be explained in the following sections.\\
\cref{sec:NeuralNetworks(BigSection)} discusses the workings of neurons and how they work together to form neural networks. \cref{sec:NeuralNetworkTraining} covers how neural networks are trained by discussing how to structure datasets, define loss functions and optimize network behavior.

	\section{Neural networks}\label{sec:NeuralNetworks(BigSection)}
	\subsection{Neurons}
The term \textbf{neural network} (NN) is a reference to the workings of nervous systems of humans \cite{NeuralNetworkLiteratureReview}. These systems consist of a net of neurons, biological cells that are intricately connected to other neurons through structures called synapses \cite{PrinciplesOfBrainFunctioningHaken}. These connections carry electric pulses between neurons that can excite them when those pulses exceed certain thresholds. Those thresholds vary from neuron to neuron and change over time. Upon excitation, new pulses in turn propagate from the excited neuron outwards to possibly excite other neurons. This interplay between excitation and transmission through the network of neurons may create what we perceive as thinking.\\
In attempts to eventually understand and replicate this thinking process, mathematical analyses of such systems  have been done as early as the 1940's \cite{A_logical_calculus_of_the_ideas_immanent_in_nervous_activity}. The artificial neural networks used in machine learning today are mathematical concepts that replicate the transmission of excitation between neurons. To examine how this is achieved using mathematical functions and values instead of biological cells and electric pulses, let's look at how the artificial neural networks are built.\\
The \textbf{neurons}, which function as building blocks of artificial neural networks, are mathematical entities that take a fixed number of scalar values as inputs and convert them into a single output value. For a more visual explanation let's take a look at \cref{fig:Neuron_explanation}, which illustrates the example of a neuron with 2 inputs.
\begin{figure}
	\centering
	\begin{tikzpicture}[shift={(0,0)}]
	\draw (5,5) circle[radius=2];
	
	% Calculate the coordinates on the circle's circumference
	\pgfmathsetmacro\arrowOneX{5 + cos(155) * 2} % 45 degrees angle
	\pgfmathsetmacro\arrowOneY{5 + sin(155) * 2}
	
	\pgfmathsetmacro\arrowTwoX{5 + cos(-155) * 2} % 135 degrees angle
	\pgfmathsetmacro\arrowTwoY{5 + sin(-155) * 2}
		
	\draw[->, thick] (1,6.5) -- (\arrowOneX, \arrowOneY) node[pos=0.5,above] {$\textcolor{orange}{\omega_1}$};
	\draw (1,6.5) node[left] {$\textcolor{blue}{a_1}$};
	\draw[->, thick] (1,3.5) -- (\arrowTwoX, \arrowTwoY) node[pos=0.5,below] {$\textcolor{orange}{\omega_1}$};
	\draw (1,3.5) node[left] {$\textcolor{blue}{a_2}$};
	\node at (5,5) {$\textcolor{OliveGreen}{e} = \textcolor{orange}{\omega_1}\textcolor{blue}{a_1} + \textcolor{orange}{\omega_2}\textcolor{blue}{a_2} + \textcolor{orange}{b})$};	
	\draw[->, thick] (7,5) -- (8.5,5) node[right] {$\textcolor{OliveGreen}{f(e)}$};
	
	
\end{tikzpicture}
	\caption{This figure aids the explanation of the operating principle of neurons in neural networks. The weights $\omega_i$ and the bias $b$ are denoted in orange, the input activations $a_i$ in blue and the activation $e$ with its corresponding activation function $f$ in magenta.}
	\label{fig:Neuron_explanation}
\end{figure}
The big circle in the middle represents the neuron itself. It takes the activation values $a_i$ as input, multiplies them with their corresponding weights $\omega_i$, sums them up, and adds a bias value $b$ to obtain the excitation $e$. It then applies the activation function onto $e$ to obtain the resulting output value. The input activations $a_i$ correspond to the strength of electronic pulses in the nervous system. The absence of a pulse in the biological system would be represented by an activation of zero in the mathematical model. The weights $\omega_i$ are a representation of how important single input values are for the activation of the neuron. In the biological counterpart this might correspond to how thick or conductive the connections between the nerve cells are. Finally, the combination of bias $b$ and activation function determines how large the sum of the input-weight-pairs has to be to activate the neuron and how the resulting value for the activation of the neuron changes for higher input activations. For example, a simple output activation function would be the ReLU function (Rectified Linear Unit).
This function is defined as \cite{ActivationFunctionOverview}
\begin{equation}
	f(x) = 
	\begin{cases}
		x, &\text{if } x\geq0 \\
		0, &\text{otherwise}
	\end{cases}.
\end{equation} 
When applying a ReLU activation function, the neuron is activated as soon as the sum of the input-weight-pairs is larger than $-b$. Its value increases linearly with $e$.
\subsection{Neural networks}\label{sec:NeuralNetworks}
A neural network can be built from these neurons by connecting the outputs of neurons to the inputs of others. An example of such a network is illustrated in \cref{fig:Neural_network_example}. This neural network consists of three layers of four neurons each, takes two values as input and outputs one value. For example, it could be used as an approximator of whether a point on a 2D-grid is inside or outside of a given region. The input values would be the $x$ and $y$ coordinates of the point and the output value could represent the predicted probability that the point is inside this region. How to find parameters that make the network correctly classify a desired region will be explained in \cref{sec:NeuralNetworkTraining}.\\
\begin{figure}
	\centering
	\begin{tikzpicture}
	\Vertex[x=0,y=0]{A}
	\Vertex[x=0,y=1]{B}
	\Vertex[x=0,y=2]{C}
	\Vertex[x=0,y=3]{D}
	
	\Vertex[x=2,y=0]{E}
	\Vertex[x=2,y=1]{F}
	\Vertex[x=2,y=2]{G}
	\Vertex[x=2,y=3]{H}
	
	\Vertex[x=4,y=0]{I}
	\Vertex[x=4,y=1]{J}
	\Vertex[x=4,y=2]{K}
	\Vertex[x=4,y=3]{L}
	
	\Edge[lw=1pt](A)(E)
	\Edge[lw=1pt](A)(F)
	\Edge[lw=1pt](A)(G)
	\Edge[lw=1pt](A)(H)
	
	\Edge[lw=1pt](B)(E)
	\Edge[lw=1pt](B)(F)
	\Edge[lw=1pt](B)(G)
	\Edge[lw=1pt](B)(H)
	
	\Edge[lw=1pt](C)(E)
	\Edge[lw=1pt](C)(F)
	\Edge[lw=1pt](C)(G)
	\Edge[lw=1pt](C)(H)
	
	\Edge[lw=1pt](D)(E)
	\Edge[lw=1pt](D)(F)
	\Edge[lw=1pt](D)(G)
	\Edge[lw=1pt](D)(H)
	
	\Edge[lw=1pt](E)(I)
	\Edge[lw=1pt](E)(J)
	\Edge[lw=1pt](E)(K)
	\Edge[lw=1pt](E)(L)
	
	\Edge[lw=1pt](F)(I)
	\Edge[lw=1pt](F)(J)
	\Edge[lw=1pt](F)(K)
	\Edge[lw=1pt](F)(L)
	
	\Edge[lw=1pt](G)(I)
	\Edge[lw=1pt](G)(J)
	\Edge[lw=1pt](G)(K)
	\Edge[lw=1pt](G)(L)
	
	\Edge[lw=1pt](H)(I)
	\Edge[lw=1pt](H)(J)
	\Edge[lw=1pt](H)(K)
	\Edge[lw=1pt](H)(L)
	
	\Vertex[x=-2,y=1, label=$x$, opacity = 0]{X}
	\Vertex[x=-2,y=2, label=$y$, opacity = 0]{Y}
	\Edge[lw=1pt](X)(A)
	\Edge[lw=1pt](X)(B)
	\Edge[lw=1pt](X)(C)
	\Edge[lw=1pt](X)(D)
	\Edge[lw=1pt](Y)(A)
	\Edge[lw=1pt](Y)(B)
	\Edge[lw=1pt](Y)(C)
	\Edge[lw=1pt](Y)(D)
	
	\Vertex[x=6,y=1.5, label=$p$, opacity = 1]{P}
	\Edge[lw=1pt](I)(P)
	\Edge[lw=1pt](J)(P)
	\Edge[lw=1pt](K)(P)
	\Edge[lw=1pt](L)(P)
	
	\draw [decorate,
	decoration = {brace, mirror, amplitude=10pt}] (-0.2,-0.5) --  (4.2,-0.5);
	\node at (2,-1.2) {3 (hidden) layers};
	\draw [decorate,
	decoration = {brace, mirror, amplitude=10pt}] (6.5,-0.3) --  (6.5,3.3);
	\node at (7.2,1.5) [rotate=-90] {width of 4 neurons};
	
\end{tikzpicture}
	\caption{This figure shows an example of a neural network. It consists of multiple neurons connected to each other. This network takes in 2 input values and returns one output value. It consists of 3 hidden layers, each having a width of 4 neurons.}
	\label{fig:Neural_network_example}
\end{figure}
This network serves as an illustrative example, showing what a neural network can look like. For real-world applications, there are various kinds of networks used to learn different tasks \cite{DeepLearningTaxonomy}. For this thesis we will only talk about "fully connected" or "dense" neural networks. This means that every neuron in the first layer will receive every possible network input value, and every neuron in later layers will receive the output of every neuron in the previous layer as input. All neurons are equipped with the same activation function. The weights and biases vary throughout. How the output of the network gets handled may still differ through different use cases. \\
When working on classification tasks, it can be beneficial to feed the output of the neural network into a so-called softmax function before analyzing it. This function converts the scalar outputs of the network, which can be any number from $\mathbb{R}$, into probabilities of the inputs belonging to specific classes. These aren't real probabilities, since every input\footnote{The term input can refer to single scalar values given to neurons, as well as to whole sets of scalar values given to the network.} belongs to a predefined class, but rather represent the network's confidence in the predicted classifications. For a set of outputs $z_i$, the standard softmax function is defined as 
\begin{equation}\label{eq:softmax}
	\sigma_i(z) = \frac{\mathrm{exp}(z_i)}{\sum_j \mathrm{exp}(z_j)}.
\end{equation}
The resulting values lie between $0$ and $1$ and add up to a total of $1$. A larger value of an output node results in a higher corresponding probability. For further details on the softmax function, see \cite{gao2018properties}.\\
The structure of a neural network is generally referred to as the \textbf{architecture} of the network, with the hidden layers\footnote{We also sometimes simply refer to hidden layers as layers.} referring to the columns of neurons in between the input values and output neurons and the amount of neurons per layer referred to as the "width" of the network. This is also denoted in \cref{fig:Neural_network_example}. 
\subsection{Mathematical view}
The previous explanations have been very visual and step by step to make the topic more accessible. However, these concepts can be broken down to rather short mathematical expressions.\\
To start off, we can define the inputs as $a_i^{(0)}$ , $i =  1, \ldots, n$. The weights of the first hidden layer can be denoted by $\omega_{i}^{(1)}$, $i =  1, \ldots, n$. Furthermore, we define $\omega_{i,j}^{(k)}$ as the weight that connects the $i$-th neuron in the $k$-th layer to the $j$-th neuron in the $(k-1)$-th layer. The maximum values of $i$ and $j$ depend on the widths of the respective layers, $k$ can reach values between $1$ and the amount of hidden layers plus 1. The bias of the $i$-th neuron in the $k$-th layer is denoted as $b_i^{(k)}$. Using this notation we can write out the output of the $i$-th neuron in the $(k+1)$-th layer as \cite{NeuralNetworksBook}
\begin{equation}
	a_i^{(k+1)} = f\left(\sum_j \left(\omega_{i,j}^{(k+1)}a_j^{(k)}\right) + b_i^{(k+1)}\right).
\end{equation}
To actually calculate this value, the activations $a_j^{(k)}$ have to be recursively replaced with their full calculation until one arrives at the input values of the network.\\
For simplicity reasons, we will refer to the weights $\omega_{i,j}^{(k)}$ and biases $b_i^{(k)}$ together as \textbf{parameters} of the neural network. We will denote these collected in one ordered set as $\theta = \{\theta_i\}_{i=1}^{N}$, where $N$ is the total number of weights and biases added together. The mapping of the parameters onto $\theta$ can be arbitrarily chosen. When talking about all parameters as a single set, the corresponding mapping has to be known, so that it's possible to calculate the output of the network when given the parameters in the same way as when given the actual weights and biases. Another possible representation that we will use often is to write this set as a vector $\theta \in \mathbb{R}^N$. A short notation for the output of the neural network will be explained in the next section.

	\section{Training of neural networks}\label{sec:NeuralNetworkTraining}
	In the previous chapter we discussed how neural networks are built up. Now we will give an example of how they can be trained and used to perform specific tasks. Specifically, we will look at how they can be trained on data sets in a process called supervised learning. For this we fix the architecture of our neural network, which includes the organisation of neurons, the input and output handling and the activation function. The things we can vary during training to make our network perform better are the parameters.

\subsubsection{Datasets}
First we need to assume that we have a given dataset containing input-output pairs that represent the task we want the neural network to perform. The neural network is supposed to learn which output is supposed to be generated from which input by evaluating the given inputs and desired outputs. We will call those desired outputs "targets" to distinct them from the actual output of the neural network during training. A good example of this would be the MNIST database, which was first used in \cite{firstMNISTpaper} in 1994 as a modification of an earlier database, and has since become a popular entry-level classification task for machine learning. Examples of input data for this database are shown in \cref{fig:MNIST_examples}. 
\img{text/MachineLearningBasics/plots/mnist_plot.pdf}{14cm}{Displayed in this figure are 4 examples of input values for the MNIST dataset. This dataset represents handwritten digits from 0 to 9. Each input value consists of a 28 by 28 grid of grayscale pixel values.}{fig:MNIST_examples}

This dataset consists of various handwritten digits from 0 to 9 represented by a 28 by 28 grid of grayscale pixel values. The values of the pixels in these grids act as the 784 input values for the neural network. The targets should be shaped according to the output of the neural network. For example one might use a scalar output of the neural network that should be equal to the value of the digit in the input pixel grid. In this case the targets are scalar values from 0 to 9 corresponding to their input pictures. Another way would be to use 10 output values together with the previously mentioned softmax function (see \cref{sec:NeuralNetworks}) to recieve 9 values as probabilities for the different numbers as output. We would then change our targets to vectors with 9 entries. The entry corresponding to the handwritten digit in the input picture would be 1, every other digit would be 0.\\

We will denote the data sets as mathematical sets of input-target pairs $\{(\mathbf{x}_i, \mathbf{y}_i)\}_{i=1}^N$, where $\mathbf{x}_i$ are the input values, $\mathbf{y}_i$ are the targets. Here we called the amount of training data $N$. The mathematical dimension of $\mathbf{x}_i$ and $\mathbf{y}_i$ can vary and are dependent on the network architecture, but we will assume that they are vectors of the real numbers $\mathbf{x}_i \in \mathbb{R}^a$ $\mathbf{y}_i \in \mathbb{R}^b$. If our inputs values are organized in a different manner, for example the MNIST data being matrices of $\mathbb{R}^{28\times28}$ we can rearrange these numbers into a vector of $\mathbb{R}^{784}$ in any way we want. The only important thing is that we convert every data point into a vector in the same way.\\
Going further we will denote all the output values of the neural network for the input $\mathbf{x}_i$ by $f_\theta(\mathbf{x}_i)$. This means that we mathematically describe the whole behavior of the neural network as a function
\begin{equation}
	f: \mathbb{R}^a \times \mathbb{R}^p \rightarrow \mathbb{R}^b
\end{equation}
with $p$ being the amount of parameters in the network.

\subsubsection{Loss function}
Now we have defined what a neural network is and how the input data we want to train on is structured. In order to train our networks to behave according to our data set, we now need to introduce a way to measure how well our network is performing. Once we can evaluate the performance of our network, we can then introduce ways to optimize that performance.\\
This measure of the performance of a neural network is called the "loss function" $\mathscr{L}$. It is also sometimes referred to as the cost function in literature. For this study we will only consider loss functions of the form 
\begin{equation}
	\mathscr{L}\left( \{(\mathbf{x}_i, \mathbf{y}_i)\}_{i=1}^{N}, \theta \right) = \frac{1}{N} \sum_{i=1}^{N} \ell_\theta\left(\mathbf{x}_i,\mathbf{y}_i\right).
\end{equation}
As a reminder, $\theta$ is a set containing the parameters of the neural network. This is the variable of the loss function that describes the neural network. How one calculates the output of the neural network using the inputs and $\theta$ has to be intrinsically defined as well. We don't denote the architecture into the dependencies of the loss function. \\
How exactly the loss function should be defined cannot be generally stated. The only thing certain is that the value of the loss function should generally be smaller the closer the output of the neural network is to the corresponding targets.\\
In most of our cases we use loss functions with 
\begin{equation}
	\ell_\theta \left( \mathbf{x}_i,\mathbf{y}_i\right) = 
	d\left(f_\theta(\mathbf{x}_i), \mathbf{y}_i\right),
\end{equation}
where $d$ represents the distance between the output vector of the neural network and the target vector according to different metrics. One very simple example would be 
\begin{equation}
	\mathscr{L}\left( \{(\mathbf{x}_i, \mathbf{y}_i)\}_{i=1}^{N}, \theta \right) = \frac{1}{N} \sum_{i=1}^{N} \sum_{j=1}^{N} \left(f_\theta(\mathbf{x_{i}})_j - y_{i,j}\right)^2,
\end{equation}
with $f_\theta(\mathbf{x_{i}})_j$ and $y_{i,j}$ being the $j$-th component of the network output and target vectors. Further examples and more general loss functions can be seen in \cite{LossExamplePaper}.

\subsubsection{Optimization}
To quickly recap, we have now defined what a neural network is and that we want a fixed network architecture during training whose parameters we can vary to optimize the performance. We also assume that we have a data set that we want our network to behave like and a loss function that measures how well the current setup of our network is performing. We will now talk about how we can change the parameters in an attempt to optimize performance, which is achieved by lowering the value of the loss function.\\
The simplest and most common algorithm to do this is stochastic gradient descent. Here, rule for updating the parameters looks like 
\begin{equation}
	\theta' = \theta - \eta \nabla_\theta \mathscr{L}\left[ \{(\mathbf{x}_i, \mathbf{y}_i)\}_{i=1}^{N}, \theta \right],
\end{equation}
where $\eta$ is a parameter of training called the "learning rate", which is a scalar value for stochastic gradient descent but could also vary through the steps and even be a tensor for more advanced optimization methods.\\

	\section{Which assumptions are actually necessary}
	In the previous sections we took a brief look at machine learning and neural networks. Although this barely scratches the surface of the methods that are used today, it's still more than what's needed for the upcoming chapters. The specific methods were given to explain how the NNs that will come up later in this thesis were trained, but are unnecessary restrictions that don't need to be made for the mathematical considerations coming up.\\
For those it is sufficient to view systems $f_\theta(\mathbf{x}_i)$ that depend on parameters $\theta$, accept input vectors $\mathbf{x}_i$ of constant dimension and can be evaluated through the formalism of the loss function from equation \cref{eq:Loss_longform}. For the discussions of the NTK, we can even disregard the assumption of the loss function splitting up into a sum of $\ell$ functions. This means that the exact properties of the neural networks we looked at earlier can be ignored, allowing us to generalize the observations to many more network architectures or completely different systems than previously mentioned.
	
	
	\chapter{Fisher Information}\label{sec:ChapterFisherInformation}
	Before starting this chapter, let's go over the ideas behind its structure.\\
First, \cref{sec:FIinStatistics} will describe the Fisher Information as it is used in statistics. Some readers may wonder why we're discussing a statistical method describing probabilities when we've only talked about machine learning and neural networks before. The answer is, that the Fisher information matrix also
acts as the Riemannian metric describing the statistical manifold of the network regarding its loss. A brief introduction to this topic will be provided in \cref{sec:Manifolds}, where we explain the concept of manifolds, \cref{sec:RiemannianMetricAndFI}, where we go over the definition of metrics on those manifolds, \cref{sec:Curvature}, where we describe the concept of curvature and \cref{sec:ApplicationOfFIToNeuralNetworks}, where we cover how these concepts apply to neural networks. Since those sections are mathematically abstract, \cref{sec:FisherInterpretation} provides a brief recap with some added intuitive explanations.\\
To conclude this chapter, \cref{sec:FIPhysics} will present another application of the Fisher Information in a physical context, where it can be used to find phase transitions in thermodynamic systems.
	\section{Use in Statistics}\label{sec:FIinStatistics}
	To introduce the Fisher information (FI), we will start off with how it's defined and used in statistics.\\
We will consider a \textbf{statistical model} $f(x_i|\theta)$ that represents how a parameter $\theta$ is related to the outcomes $x_i$ of random variables $X_i$ \cite{StatisticFisherInfoTutorial}. Let's look at an example of a statistical model. You can see a picture of a Galton board in \cref{fig:GaltonPicture}.
\begin{figure}
	\centering
	\includegraphics[width = 4cm]{text/FisherInformation/plots/GaltonBoard.jpg}
	\caption{Photograph of a Galton board, taken from \cite{GaltonBoardPicture}.}
	\label{fig:GaltonPicture}
\end{figure}
It's a famous mechanical model that visualizes binomial distributions, which are discrete approximations of the normal distribution. If we place many balls at the top of the board and let them fall to the bottom, the amount of balls that end up in each cell are distributed according to the binomial distribution \cite{GaltonBoardArticle}. In this case, $x_i$ could assume the slot number which a ball can fall into. The $i$ could label multiple throws into the board, but for now we'll assume that there is only one experiment $i$. To introduce a parameter that influences the distribution of the balls, let's say one can throw from different spots above the Galton board which we now control with the the value of $\theta$. For a known $\theta$, the resulting function of the statistical model represents the probability distribution $f(x_i|\theta) = p_\theta(x_i)$ for the probability of the different outcomes $x_i$. As a sidenote, if we instead fixed the value of $x_i$ and viewed the statistical model as a function of $\theta$ it would be called a likelihood function. A visual representation of the probability distributions for different $\theta$ can be seen in \cref{fig:GaltonDistributions}.
\begin{figure}
	\centering
	\includegraphics[width = \textwidth, clip, trim= 0cm 0cm 0cm 2.3cm]{text/FisherInformation/plots/GaltonDistributionsPlot.pdf}
	\caption{This figure shows the probability distributions for a Galton board with different drop-in positions. The slots where the ball can end up are labeled by the value of $x_i$.}
	\label{fig:GaltonDistributions}
\end{figure}\\
In general, the statistical models might be more complex, where $\theta$ contains several parameters, $x$ is an element of a mathematical space other than $\mathbb{R}$ and the index $i$ denotes various different experiments, all depending on the same parameter but having different possible outcomes and probability distributions.\\
What's of interest for the FI are cases, where the parameters are not known before conducting the experiment and have to be approximated by the different outcomes $x_i$.\\
Before we introduce the FI, let's look at an example from Ly et al. \cite{StatisticFisherInfoTutorial}. Let's consider a biased coin where we denote the probability of heads ($x_i = 1$) with $\theta$ and the probability of tails ($x_i = 0$) with $1-\theta$. We will now take a look at the outcome of $n$ tosses, represented by the variable $X^n$. For example, an observed result for $X^5$ could be $x^5 = (1,1,0,1,0)$. Let's consider another variable $Y$, observing the sum of the total head throws $y = \sum x^n$. In our example case of $x^5 = (1,1,0,1,0)$, this would result in a value of $y = 3$. The probability for this variable $y$ is distributed according to the binomial distribution $f(y|\theta) = \binom{n}{y}\theta^y (1-\theta)^{n-y}$ \cite{BookOnBinomialDistributions}. Here, the binomial coefficient $\binom{n}{y}$ represents the different combinations that result in the same value of $y$. This is needed because there are $2^n$ different possibilities for $x$, while there are only $n$ different possibilities for $y$.\\
If we now fix the outcome of $y$ and look at the conditional probability for the different $x^n$ that could have resulted in that $y$ value, we get $p(x|y,\theta) = 1/ \binom{n}{y}$. With $p(x|y,\theta)$ we denote the probability depending on $x$ for fixed $y$ and $\theta$. Although the probability of $y$ and $x$ both depend on $\theta$, the probability for $x$ when $y$ is fixed doesn't. After measuring $y$, there is no information about $\theta$ left in the measurement of $x$. This means that $y$ is fully descriptive of, or sufficient for the parameter $\theta$. Measuring $y$ results in the same amount of information about the parameter $\theta$ as measuring the whole observation $x$. To quantify how much information a certain function contains about the parameters $\theta$, Fisher introduced the \textbf{Fisher information}.\\
The Fisher information is defined as 
\begin{equation}\label{eq:FIDefinition}
	I_{X,ij}(\theta) = \underset{x\in X}{E} \left[\tAbl{}{\theta_i}\log f(x|\theta) \cdot \tAbl{}{\theta_j}\log f(x|\theta)\right],
\end{equation}
where we used the expectation $E$
\begin{equation}
	\underset{x\in X}{E} \left[A(x)\right] = 
	\begin{cases}
		\sum_{x\in X} \left(A(x) p(x)\right) &\text{if $X$ is discrete},\\
		\int_{x\in X} A(x) p(x) \mathrm{d}x &\text{if $X$ is continuous}.
	\end{cases}
\end{equation}
We will later use an alternative notation where we denote $\log f$ as $\ell$. As will be evident later, this notation does not interfere with the definition of $\ell$ in \cref{eq:Loss_longform}. Since the Fisher information is dependent on $\theta$, we can fix the value of $\theta$ during calculation, which makes $f(x|\theta)$ equal to the probability density $p_\theta(x)$.\\ 
For $n$ independent experiments $X^n$, where $f(x^n|\theta) = \prod_{i=1}^n f(x_i|\theta)$, one can split the FI into 
\begin{equation}\label{eq:FIforIndependentExperiments}
	I_{X^n,ij}(\theta) = \prod_{i=1}^n I_{X_i,ij}(\theta).
\end{equation}
A proof of this can be found in \cref{sec:ProofFIforIndependentExperiments}.\\
For our example, the FI yields $I_{X^n}(\theta) = I_{Y}(\theta) = n/(\theta(1-\theta))$ \cite{StatisticFisherInfoTutorial}. This means that there's as much information about the $\theta$ contained in the measurement of $Y$ as in the measurement of $X^n$, which coincides with $Y$ being a sufficient measurement for $\theta$. \\
To give another example of how the FI represents the information obtainable about a parameter from a measurement, let's consider the family of normal distributions
\begin{equation}
	\mathcal{N}(x|\mu,\sigma) = \frac{1}{\sqrt{2\pi\sigma^2}}\mathrm{e}^{-(x-\mu)^2/(2\sigma^2)}.
\end{equation} 
These will now act as our statistical model $p_\theta(x) = \mathcal{N}(x|\theta)$, where $\theta = \{\theta_1,\theta_2\} = \{\mu, \sigma\}$. An observation would consist of a resulting value $x\in \mathbb{R}$, with its probability distributed according to the statistical model. The FI from equation \cref{eq:FIDefinition} can be derived as 
\begin{equation}
	I(\mu,\sigma) = \frac{1}{\sigma^2}
	\begin{pmatrix}
		1 & 0 \\
		0 & 2
	\end{pmatrix}.
\end{equation}
We can now interpret the diagonal elements as measures of how much information a measurement contains about the corresponding parameter, and the off-diagonal elements as measurements of how similarly the model changes when varying the corresponding parameters. To give a specific example, let's look at the diagonal component corresponding to $\mu$, $I_{11}(\mu,\sigma) = 1/\sigma^2$. This value indicates that for smaller $\sigma$, random values drawn from the distribution contain more information about $\mu$ than samples drawn from distributions with larger $\sigma$. For a visual guide, consider \cref{fig:NormalDistributionExample}.
\begin{figure}
	\centering
	\includegraphics{"text/FisherInformation/plots/NormalDistributionPlot.pdf"}
	\caption{This figure shows two normal distributions centered around $\mu = 2$ with varying $\sigma$ parameters. It also shows four samples chosen randomly according to the distribution. It's visible that for the case of a smaller variance $\sigma$, the points tend to be closer to the center and also less spread apart, which makes the information about $\mu$ contained in a measurement larger for a smaller variance.}
	\label{fig:NormalDistributionExample}
\end{figure}
It can be seen that the smaller value of $\sigma$ results in a narrower spread of randomly drawn values, as indicated by the orange arrow. The values also tend to be closer to the mean value $\mu$ the smaller the variance $\sigma$ is. Therefore, if we had to predict the value of $\mu$ from knowing only a few drawn samples, it would be easier to use the values drawn from the distribution with the smaller variance. This is because the information contained in the samples measured by the FI is greater there. Keep in mind that all of these drawn values are randomly distributed. Therefore it's also possible to have two sets of random samples where the samples from the larger variance are better at predicting $\mu$, but statistically speaking the smaller variance tends to perform better.

To conclude this chapter, the Fisher information is used in statistics to measure the amount of information one can gather about a parameter $\theta$ by measuring the outcome of a probability distribution $p_\theta(x_i)$. It is defined in \cref{eq:FIDefinition}.

	\section{Fisher Information as Riemannian metric}
	Previously we talked about the FI in the context of statistics. You may wonder why we went over a statistical method to describe probabilities, when we previously only talked about machine learning and neural networks. The answer to this question is, that the Fisher information matrix defined as 
\begin{equation}
	I_{ij} = \underset{(\mathbf{x}_i,\mathbf{y}_i) \in D}{E} \left[\tAbl{}{\theta_i}\ell(\mathbf{x}_i,\mathbf{y}_i)\cdot \tAbl{}{\theta_j}\ell(\mathbf{x}_i,\mathbf{y}_i)\right]
\end{equation}
acts as the Riemannian metric describing the statistical manifold of the network regarding its loss. A brief introduction to this topic will be provided in this chapter. Keep in mind that this is mostly for intuition purposes and we will only cover a few important definitions. For more details, please refer to \cite{AmarisLectureNotes} where most of the following information is taken from. If you're not interested in mathematical details you can skip this chapter and go directly to \textbf{FILL SOMETHING IN HERE}.

\subsection{Differentiable manifolds}
To state the definition, a $n$-dimensional manifold $S$ is a topological space so that for every point you can define a neighborhood around that point which is homeomorphic to an open subset of $\mathbb{R}^n$. A good example of this would be the surface of the earth, where locally viewed in the scale that we usually see things, the earth appears flat, but on a global scale the earth is obviously a sphere. This results for example in the shortest path between to points not being a straight line in maps of the world as a whole. Also the angles of a triangle don't sum up to \SI{180}{\degree} as they would in a subspace of $R^n$. This results from conventional maps being subspaces of $\mathbb{R}^2$, although the earth is only homeomorphic to $\mathbb{R}^2$ in smaller local scales. If one tries to map the whole sphere into a map without gaps, one has to map the coordinates in a way that makes the shortest lines curved for example.\\
Let's come back to the statistical models $f(x|\theta)$ we talked about in the last section. We will treat the models considering fixed parameters as probability distributions $p_\theta(x)$ in this context. If the probabilities are sufficiently smooth in $\theta$, which means that they are differentiable in $\theta$ as often as needed for further considerations, one can view the family of probabilities as a $n$-dimensional manifold, where the $n$ different $\theta$ components play the role of the coordinate system of the manifold. \\
For example let's consider normal distributions 
\begin{equation}
	p(x|\mu,\sigma) = \frac{1}{\sqrt{(2\pi\sigma^2)}} \mathrm{e}^{-(x-\mu)^2/(2\sigma^2)},
\end{equation}
where $\theta = \{\theta_1,\theta_2\} = \{\mu,\sigma\}$. We can now consider this family of distributions as a manifold, displayed in \cref{fig:NormalDistributionManifold}. This is like a space, where every point in the space represents a distribution $p(x|\theta)$.

\begin{figure}
	\centering
\begin{tikzpicture}
	% Draw horizontal lines
	\foreach \y in {0,1,2,3,4}
	\draw (-4.2,\y) -- (4.2,\y);
	
	% Draw vertical lines
	\foreach \x in {-4,-3,-2,-1,0,1,2,3,4}
	\draw (\x,0) -- (\x,4.2);
	
	% Draw x and y axes
	\draw[thick, ->, line width=1.5pt] (0,0) -- (4.5,0) node[below] {$\mu$};
	\draw[thick, ->, line width=1.5pt] (0,0) -- (0,4.5) node[above] {$\sigma$};
	
	% Label tick marks
	\foreach \x in {-4,-3,-2,-1,0,1,2,3,4}
	\draw (\x,-0.1) -- (\x,0.1) node[below=3pt] {\x};
	\foreach \y in {1,2,3,4}
	\draw (0,\y) -- (-0.2,\y) node[below] {\y};
	
	\draw[thick, ->, >= latex, line width=1.5pt] (2.7,4.5) -- (2,3);
	\node at (2.7,4.5) [above] {$p(x|\mu = 2,\sigma = 3)$};
		
\end{tikzpicture}
\caption{This figure illustrates the manifold of normal distributions. As coordinate system, $\mu$ and $\sigma$ are used. Every point in this manifold represents a probability distribution, as indicated by the arrow. \label{fig:NormalDistributionManifold}}
\end{figure}
It might also be clear that the coordinate system of a manifold is definable in multiple different ways. Although it's always given in our use case, let's therefore denote that in general when we have coordinates $\theta$ we also need a mapping $\phi$, which maps coordinates to points on a manifold. This means that by applying $\phi(p)$ to a point $p\in S$ the resulting vector in $\mathbb{R}^n$ resembles the coordinates of that point. We can also apply the inverse of that mapping to a set of coordinates to the point in the manifold that's represented by those coordinates.
%Now we need to introduce some assumptions for the following theorems and definitions:
%\begin{enumerate}
%	\item All $p(x|\theta)$ must have a common "support" $X$ so that $p(x|\theta)>0$ for all $x\in X$.
%	\item Let's define $\ell(x|\theta)$ = $\log p(x|\theta)$. For every fixed $\theta$, the $n$ functions $\partial/\partial \theta_i \ell(x|\theta)$ labeled by $i$ have to be linearly independent. We will later see that $\ell$ here corresponds to the $\ell$ of the loss function if we consider the manifold of a machine learning network.
%	\item The moments of random variables $\partial/\partial \theta_i \ell(x|\theta)$ exist up to necessary orders. 
%	\item The partial derivative $\partial/\partial\theta_i$ and the integration over the measure of $X$ can always be interchanged.
%\end{enumerate}

\subsection{Tangent space}
The tangent space $T_p$ of a manifold at point $p$ is a vector space obtained by linearization of the manifold around $p$. For intuition purposes, let's take a look at the tangent plane of a 2d-surface in \cref{fig:TangentSpacePlot}. Here the tangent space is simply a plane that touches the surface in one point, with derivatives adjusted to match the surface at that point.
\begin{figure}
	\centering
	\includegraphics[width = 12cm, clip, trim= 0cm 1.5cm 0cm 2cm]{text/FisherInformation/plots/TangentSpacePlot.pdf}
	\label{fig:TangentSpacePlot}
	\caption{This figure contains an example of a tangent space of a manifold, which in this case is a 2d-surface.}
\end{figure}
For the general case of $n$-dimensional manifolds, it is obvious that tangent spaces aren't simply tangent planes of surfaces in every case, therefore let's introduce a way how to calculate a tangent space.\\
First we will define curves $c(t)$ that are continuous mappings from an interval $[a,b] \in \mathbb{R}$ into the manifold $S$. In the parametric representation, the curve is given by $\theta(t)$. Now we can define what a tangent vector is.\\
Imagine a smooth real function $f(\theta): S \rightarrow \mathbb{R}$. We can now restrict this function to our predefined curve $c$ by $f \circ c : [a,b] \rightarrow \mathbb{R}$. We'll denote this via $f\left(\{\theta(t)\}\right)$ in the coordinate expression. The derivative $Cf$ of this function is then given by 
\begin{equation}
	Cf = \tAbl{f\circ c}{t} = \sum_{i=1}^{n} \pAbl{f}{\theta_i}\tAbl{\theta_i(t)}{t} = \underbrace{\left(\sum_{i=1}^{n}\tAbl{\theta_i}{t}\pAbl{}{\theta_i}\right)}_{\text{Operator }C} f.
\end{equation}
Therefore we can associate a directional derivative operator $C$ with each curve. The only dependence of this operator regarding the curve is $\tAbl{\theta_i}{t}$, where we might also clarify that this derivative depends itself on the point where it is calculated at.\\
If the manifold is infinitely differentiable, the set of these mappings $C$ at a fixed point on the manifold forms a $n$-dimensional vector space, called the "\textbf{tangent space}" $T_p$ of that point $p$. \\
To make this more clear, let's look at the simplest basis for this vector space, which we'll call the "natural basis" of the tangent space. For this we'll consider curves $c_1,c_2, \dotsc c_n$ through a point $p_0$, where the curves are defined as 
\begin{equation}
	c_i(t) = \{\theta_1^0,\theta_2^0, \dotsc, \theta_i^0 + (t-t_0), \dotsc \theta_n^0 \},
\end{equation}
so that $c_i(t_0) = \{\theta_1^0,\theta_2^0, \dotsc\theta_n^0\} = p_0$ for every curve. The tangent vectors $C_i$ then are simply the derivatives regarding their corresponding coordinate $C_i f = \partial/\partial \theta_i f$. We will denote this in short by $C_i = \partial_i$. The $n$ vectors $\partial_i$ are linearly independent and form the natural basis for the tangent space. This means that any tangent vector $A$ can be represented by 
\begin{equation}
	A = \sum_{I=1}^{n} A_i \partial_i,
\end{equation}
with components with respect to the natural basis $A_i$. If we are given a curve $c(t)$ going through $c(t_0)$ which derivative operator $C$ in $t_0$ is equivalent to the vector $A$, we can find the natural basis representation of $A$ as
\begin{equation}
	A_i = \left. \tAbl{\theta_i}{t}\right|_{t_0}.
\end{equation} \\
Now let's take a look at the case of manifolds of statistical models. First we will define $\ell$ as
\begin{equation}
	\ell(x|\theta) = \log f(x|\theta),
\end{equation}
and assume that for fixed $\theta$, the $n$-functions $\partial_i \ell(x|\theta)$ are linearly independent. This means that we can construct a vector space by defining
\begin{equation}
	T_\theta^{(1)} = \{A(x) | A(x) = A_i \partial_i \ell(x|\theta)\}.
\end{equation}
We do this because there is a natural isomorphism between the tangent space $T_\theta$ and this space $T_\theta^{(1)}$ through 
\begin{equation}
	\partial_i \in T_\theta \leftrightarrow \partial_i \ell(x|\theta) \in T_\theta^{(1)}.
\end{equation}
We will call $T_\theta{(1)}$ the "\textbf{1-representation}" of our tangent space for the statistical models. We will now use this vector to define an inner product on the tangent space and it's 1-representation.\\
\subsection{Riemannian metric and Fisher Information}
When the inner product of the tangent spaces $T_p$ is defined, the manifold is called a \textbf{Riemannian manifold}.\\
Let's first consider the inner product of the 1-representation space. Let $A(x)$ and $B(x)$ be 1-representations of $A$ and $B \in T_\theta$. It is intuitive to define the inner product as 
\begin{equation}
	\langle A(x), B(x) \rangle = \underset{x\in X}{E} \Big[A(x) B(x)\Big],
\end{equation}
with the expectation value $E[\cdot]$. Since the tangent space is isomorphic to its 1-representation, the inner product also translates via 
\begin{equation}
	\langle A, B \rangle = \langle A(x),B(x) \rangle.
\end{equation}
This also means that we can calculate the inner product of the basis vectors as 
\begin{equation}
	\begin{split}
		g_{ij}(\theta) \vcentcolon= \langle \partial_i, \partial_j\rangle = \langle \partial_i\ell(x|\theta), \partial_j\ell(x|\theta)\rangle \\
		= \underset{x \in X}{E} \Big[\partial_i\ell(x|\theta), \partial_j\ell(x|\theta)\Big].
	\end{split}
\end{equation}
The resulting object $g_{ij}(\theta)$ is called the \textbf{Riemannian metric tensor} of the manifold. We can see that the Riemannian metric tensor for the statistical model is equivalent to the Fisher Information. It might be of interest to note here that we assume that $\ell$ only depends explicitly on $\theta$. If we denote these as implicit dependencies, we have to replace the partial derivatives with absolute ones.\\
The inner product of two vectors can now be expressed with the metric tensor as
\begin{equation}
	\langle A,B \rangle = \sum_{i,j} A_iB_jg_{ij}(\theta)
\end{equation}
in the component form. \\
Using this representation of the inner product, we can define various things. For example, the length of a vector $A$ is defined as $|A|^2 = \sum_{i,j} A_iA_j g_{ij}$, the orthogonality of two vectors when their inner product is zero, and the distance between two points $\theta^{(0)}$ and $\theta^{(1)}$ along the curve $c$ is defined by 
\begin{equation}
	s = \int_{t_0}^{t_1} \sum_{i,j} \sqrt{g_{ij}\tAbl{\theta_i}{t}\tAbl{\theta_j}{t}} \mathrm{d}t. 
\end{equation}
This also introduces the concept of \textbf{Riemannian geodesics}. These are the curves that connect two points via the minimal distance between the two.\\
Another representation of the metric tensor or the FI is 
\begin{equation}\label{eq:SecondRepresentationOfFisherInfo}
	g_{ij}(\theta) = - \underset{x\in X}{E} \Big[ \partial_i \partial_j \ell(x|\theta) \Big].
\end{equation}
A proof of this is denoted in \cref{sec:ProofForeq:SecondRepresentationOfFisherInfo}.

\subsection{Scalar curvature and Christoffel symbols}
Here we will only give the definitions needed to compute the scalar curvature of a statistical manifold. A full understanding requires much more mathematics than we will go over here. If you are interested in this, please see \cite{AmarisLectureNotes}, \cite{PaperOnCurvature} and \cite{GeneralRelativityBook}.\\
If we use Einstein notation, the scalar curvature is given by 
\begin{equation}
	S = g^{\mu \nu} \left(\Gamma^{\lambda}_{\mu \nu, \lambda} -\Gamma^{\lambda}_{\mu \lambda, \nu}
	+\Gamma^{\sigma}_{\mu \nu}\Gamma^{\lambda}_{\lambda \sigma}
	-\Gamma^{\sigma}_{\mu \lambda}
	\Gamma^{\lambda}_{\nu \sigma}\right),
\end{equation}
with the inverse components of the Riemannian metric $g^{ij} = (g^{-1})_{ij}$ and the Christoffel symbols
\begin{equation}
	\Gamma^{i}_{jk} = \frac{1}{2}g^{im} \left(\pAbl{g_{mk}}{\theta_l} + \pAbl{g_{ml}}{\theta_k} - \pAbl{g_{kl}}{\theta_m}\right),
\end{equation}
where $\Gamma^{i}_{kl,m} = \partial_m \Gamma^{i}_{kl}$.
	\section{Intuitive explanations}\label{sec:FisherInterpretation}
	In the previous sections we've introduced Fisher Information, first in a statistical context and later as the metric of a statistical manifold, in order to apply some of its insights to neural network training.\\
Since the information provided in the last sections has been mathematically abstract and lacking intuition, let's quickly recap the basics of what we need to know and understand about the Fisher Information for the rest of this work.\\
First, let's state the definition again. The FI is defined as 
\begin{equation}
	\begin{split}
		I_{ij} &= \underset{x \in X}{E} \left[\tAbl{}{\theta_i}\log f(x|\theta)\cdot \tAbl{}{\theta_j}\log f(x|\theta)\right]\\
		&= \underset{(\mathbf{x}_i,\mathbf{y}_i)\in D}{E} \left[\tAbl{}{\theta_i}\ell_\theta(\mathbf{x}_i,\mathbf{y}_i)\cdot \tAbl{}{\theta_j}\ell_\theta(\mathbf{x}_i,\mathbf{y}_i)\right],
	\end{split}
\end{equation}
where the notation in the first line is used in the context of statistics and the one from the second line in the context of the manifold of neural network training.\\
We first introduced the FI as a statistical measure of how much information the measurement of a random variable contains about its underlying parameters.\\
For example, let's say our experiment consists of throwing a biased coin, where the probability of heads is the underlying parameter. We can conduct an experiment to estimate the probability of the biased coin. Let's say we make 10 throws twice. The first time, we will use the full observation for the parameter estimation. We will exactly know the results of the coin toss \emph{for every single one} of the 10 throws. The second time, we will only know how often heads came up \emph{in total} for the 10 throws. We will disregard information about when exactly during these 10 trials we measured heads or tails. The Fisher Information with respect to the parameter yields the same value for both of these observables. This therefore means that the information about the probability of heads contained in the measurement is the same for both observations \cite{StatisticFisherInfoTutorial}. To estimate the bias of the coin, it is perfectly sufficient to only write down the total number of heads instead of the entire observation.\\
Going further, we introduced another application of the FI as the Riemannian metric of the statistical manifold corresponding to a neural network. For this, we considered a neural network along with a dataset and loss function as a statistical model. We did this by introducing a kind of probability distribution $p_\theta(\mathbf{x}_i,\mathbf{y}_i) = \mathrm{e}^{-\ell_\theta(\mathbf{x}_i,\mathbf{y}_i)}$. This probability measures if our network with parameters $\theta$ produces the correct output $\mathbf{y}_i$ when given the input $\mathbf{x}_i$. The family of probability distributions $\{p_\theta\}$ forms a statistical manifold \cite{AmarisLectureNotes}. It is a topological space with coordinates $\theta = \{\theta_1, \ldots, \theta_n\}$, where every point in the space corresponds to a probability function that depends on the $\theta$ coordinates. Locally, it is isometric to a subset of $\mathbb{R}^n$. Globally, euclidean geometry doesn't apply, which for example makes the shortest paths between points in the manifold curves in the coordinate system. Because every point in the space is a probability function, it's very hard to properly imagine lines between points and even the seemingly simple concept of distance becomes very abstract. Therefore, our next goal was defining how one can measure distance and characterize curvature in this space.\\
To be able to define what distance means, we first introduced the concept of a tangent spaces. A tangent space is, again hard to grasp, an abstract vector space defined at a point in the manifold, where the vectors are differential operators corresponding to the derivatives along curves through the point \cite{AmarisLectureNotes}. The reason we defined it is because by defining an inner product $\langle \cdot , \cdot \rangle$ on the tangent spaces, we can obtain a metric on the manifold through $g_{ij} = \langle \mathbf{e}_i,\mathbf{e}_j\rangle$ \cite{AmarisLectureNotes}. The basis vectors $\mathbf{e}_i$ on the tangent spaces are abstract concepts, which makes finding an intuitive definition of an inner product non trivial. That's why we introduced a new space, isomorph to the tangent space, where defining the inner product feels natural. The ismorphism was introduced as \cite{AmarisLectureNotes}
\begin{equation}
	\partial_i \leftrightarrow \partial_i \ell(x|\theta)
\end{equation}
and maps the derivative operators onto actual derivatives of the logarithm of the statistical model $f$. To obtain the inner product of two derivative operators we then defined \cite{AmarisLectureNotes}
\begin{equation}
	\langle \partial_i, \partial_j \rangle = \langle \partial_i \ell(x|\theta), \partial_j \ell(x|\theta) \rangle,
\end{equation}
where we naturally assume the inner product between the two distributions as
\begin{equation}
	 \langle \partial_i \ell_\theta(\mathbf{x}_i,\mathbf{y}_i), \partial_j \ell_\theta(\mathbf{x}_i,\mathbf{y}_i) \rangle = \underset{(\mathbf{x}_i,\mathbf{y}_i)\in D}{E} \big[\partial_i \ell_\theta(\mathbf{x}_i,\mathbf{y}_i) \cdot \partial_j \ell_\theta(\mathbf{x}_i,\mathbf{y}_i)\big].
\end{equation}
Now we arrived at the definition of the metric of our manifold, which is equivalent to the Fisher Information matrix
\begin{equation}\label{eq:MetricDefinitionInInterpretationChapter}
	I_{ij} = g_{ij} = \underset{(\mathbf{x}_i,\mathbf{y}_i)\in D}{E} \big[\partial_i \ell_\theta(\mathbf{x}_i,\mathbf{y}_i) \cdot \partial_j \ell_\theta(\mathbf{x}_i,\mathbf{y}_i)\big].
\end{equation}
Since this equation is a bit easier to grasp, we can finally try to get some intuition about the connections of the concepts mentioned before.\\
Distance between two points in a curved manifold along a curve $c$ is defined as \cite{AmarisLectureNotes}
\begin{equation}
	s = \int_{t_0}^{t_1} \sum_{i,j} \sqrt{g_{ij}\tAbl{\theta_i}{t}\tAbl{\theta_j}{t}} \mathrm{d}t. 
\end{equation}
If we only move along one coordinate $\theta_i$ on the curve, the distance reduces to 
\begin{equation}
	s = \int_{t_0}^{t_1} \sqrt{g_{ii}}\left|\tAbl{\theta_i}{t} \right| \mathrm{d}t. 
\end{equation}
Therefore the diagonal components of the FI tell us how far we move across the manifold when we change parameter $\theta_i$. Through inspection of the equation for the metric (\cref{eq:MetricDefinitionInInterpretationChapter}), it's clear that the distance between points created by changing a parameter relates to the expected change in the logarithm of the probability distribution. We can directly map the parameter space onto the manifold of probabilities, but we need to reconsider the notion of distance for the manifold because distance in the euclidean parameter space doesn't generally translate to distance in the manifold, which is a measure of difference between the probabilities.\\
Finally let's examine the components of the metric for the neural network. The diagonal components are the expectation values of the squared derivative of the loss. This means that the diagonal values represent how much our subloss changes when we vary the corresponding parameter. The off-diagonal values represent how similar the change in the subloss function is under changes in the two corresponding parameters. This information about the change of the loss regarding the parameters is now of high interest when considering neural network training. In general, calculating the FI computationally is very costly though, because the number of parameters can be very high for complex tasks. That's why the results of this work consider a few observations resulting from the fisher matrix and investigate some possibilities to obtain them from other, computationally easier, observables of training instead.
	\section{Investigation of physical phase transitions using Fisher Information}
	The Fisher Information can also be useful in physical context. It is possible to use it to find phase transitions in thermodynamic systems. We will later look at the possibility of applying this to neural network training in search of processes equivalent to phase transitions, which may lead to more insight into what exactly influences training.\\
To explain how the FI can be used to find phase transitions let's briefly review \cite{Prokopenko}.\\
Given an equilibrated physical system in a large thermal heat bath, statistical models of those systems usually deal with Gibbs measures of the form 
\begin{equation}\label{eq:ThermodynamicGibbsMeasure}
	p(x|\theta) = \frac{1}{Z(\theta)} \mathrm{exp}\left(\sum_i \theta_i X_i(x)\right).
\end{equation}
Here, $x$ represent the microstates of the system, $X_i$ are time-independent functions called "collective variables" and $\theta_i$ represent the time-dependent thermodynamic variables. These $\theta_i$ could be, for example, temperature, pressure, magnetic fields etc. They will be used as parameters $\theta$ of the FI.\\
Using the thermodynamic variables one can construct thermodynamic potentials. One example used in the paper is the Helmholtz free energy 
\begin{equation}
	A = - k_B T \ln Z(\theta),
\end{equation}
where we consider $k_B T = 1/\beta$ to be one of the thermodynamic variables. \\
Now it's stated that a classification of phase transitions typically requires an examination of the derivatives of the thermodynamic potential. Specifically, there are cases where an order parameter $\phi^i$ describing the phase transition is representable as a negative derivative of the potential over some thermodynamic variable $\theta^i$. In this case, the diagonal components Fisher Information defined by the probability distributions in \cref{eq:ThermodynamicGibbsMeasure} and \cref{eq:FIDefinition} can be written as
\begin{equation}
	I_{ii} = \beta \pAbl{\phi^i}{\theta^i}.
\end{equation}
There are second order phase transitions, where the order parameter (which is a derivative of the thermodynamic potential) changes continuously while it's derivative diverges. Using this relationship, one can identify those phase transitions by searching for divergences in the diagonal components of the Fisher Information.\\
In addition to that, \cite{Janke} visits the same thermodynamic metric described by the Fisher Information. Here the scalar curvature corresponding to the metric $\mathscr{R}$ is introduced as a measure of complexity for the physical systems. It is stated that for all models that they've considered so far, $\mathscr{R}$ diverges at, and only at, the phase transition.\\
This gives us two ways of finding phase transitions which we can also apply to the Fisher Information of neural network training, to possibly search for analogues of physical phase transitions in the training.
	
	
	\chapter{Neural Tangent Kernel}\label{sec:ChapterNTK}
	%This chapter introduces the \textbf{Neural Tangent Kernel} (NTK). While the Fisher information describes the parameter derivative correlations of the loss functions, the NTK describes the similarities in the neural network output regarding the parameters.\\
\cref{sec:NTKderivation} presents a derivation of the NTK using the Gradient Flow assumption. \cref{sec:NTKInterpretation} provides further explanations on why the NTK arises in this context and how it can be interpreted.
	\section{Derivation from Gradient Flow}
	\subsection{What we call time}
A simple and intuitive way to introduce the NTK is via "\textbf{Gradient Flow}", which is an assumption related to the SGD algorithm from \cref{sec:NetworkOptimization}. To quickly recap the update step from stochastic gradient descent, it is defined as 
\begin{equation}
	\theta' = \theta - \eta \nabla_\theta \mathscr{L}\left( \{(f_\theta(\mathbf{x}_i), \mathbf{y}_i)\}_{i=1}^{N} \right).
\end{equation}
The "flow" aspect arises when we start to ignore the discrete nature of these update steps and assume that the $\theta$ parameters change continuously. A visual demonstration of how this changes the evolution of the parameters can be seen in \cref{fig:GradientFlowPlot}. Here, the dashed markers represent the evolution under regular SGD, while the solid line represents the evolution for gradient flow.
\begin{figure}
	\centering
	\includegraphics[width=12cm, clip, trim = 0cm 2.3cm 0cm 3.5cm]{text/NTK/GradientFlowPlot.pdf}
	\caption{This graph shows the effect of assuming gradient flow. The dashed line along with the circle markers show the positions during regular SGD with finite step sizes, the solid line shows the path of the parameters under gradient flow.}
	\label{fig:GradientFlowPlot}
\end{figure}
To do this, we first introduce a notion of "time" into our system. We try to visualize the optimization process as an evolution of our parameters $\theta$ through this variable called time, which converts updating the parameters into moving further along the timeline of our parameters. We now translate $\theta \rightarrow \theta(t)$ and $\theta' \rightarrow \theta(t+\Delta t)$. This time in our system doesn't work exactly the same way as physical time, but since the process of calculating better parameters and changing them is always associated with the expenditure of physical time, it is intuitive to refer to our system's propagation variable as "time". \\
Returning to the SGD algorithm, since the learning rate $\eta$ affects the size of our update step, we will refer to $\eta$ as the amount of time it takes to update a parameter $\theta' \rightarrow \theta(t+\eta)$. The whole SGD algorithm then becomes
\begin{equation}
	\theta(t+\eta) = \theta(t) - \eta \nabla_{\theta(t)} \mathscr{L}\left( \{(f_{\theta(t)}(\mathbf{x}_i), \mathbf{y}_i)\}_{i=1}^{N} \right)
\end{equation}
which we can rewrite to 
\begin{equation}
	\frac{\theta(t+\eta)-\theta(t)}{\eta} = - \nabla_{\theta(t)} \mathscr{L}\left( \{(f_{\theta(t)}(\mathbf{x}_i), \mathbf{y}_i)\}_{i=1}^{N} \right).
\end{equation}
When observing the term on the left side, readers who are familiar with calculus might recognize that it looks similar to the definition of a derivative in time
\begin{equation}
	\pAbl{}{t}\theta(t) = \lim_{\eta\rightarrow0} \frac{\theta(t+\eta)-\theta(t)}{\eta}.
\end{equation}
This means that for very small learning rates we can approximate the SGD as 
\begin{equation}\label{eq:NTKThetaDerivative}
	\pAbl{}{t}\theta(t) = - \nabla_{\theta(t)} \mathscr{L}\left( \{(f_{\theta(t)}(\mathbf{x}_i), \mathbf{y}_i)\}_{i=1}^{N} \right)
\end{equation}
with the partial derivative of $\theta$ along the assumed time variable.\\
For visual simplicity reasons, let's define the $j$-th component of the network output for the $i$-th input point $f_{\theta(t)}(\mathbf{x}_i)_j$ as $f_{ij}$ and assume Einstein summation for the rest of the chapter.
This means that when an index is occuring twice we don't denote a hidden summation over all possible values for this index (for example $a_kb_k = \sum_k a_kb_k$).\\
Because it will be convenient later we also spell out one component of the $\nabla_\theta$ derivation of the loss function further by using the chain rule as
\begin{align}
	\pAbl{}{t}f_{\theta(t)}(\mathbf{x}_i)_j = \pAbl{\theta_k}{t} &= - \pAbl{}{\theta_k}\mathscr{L}\left( \{(f_{\theta(t)}(\mathbf{x}_i), \mathbf{y}_i)\}_{i=1}^{N} \right)\nonumber\\
	&= - \pAbl{\mathscr{L}}{f_{ij}} \cdot \pAbl{f_{ij}}{\theta_k}.
\end{align} 

\subsection{Derivation of the NTK}
This notion of time affects not only the parameters, but also everything that depends on them. For example, since the network output of a fixed architecture for a given input data point only depends on the parameters of the network, it can also be mathematically viewed as dependent on the time $f_{\theta}(\mathbf{x}_i) \rightarrow f_{\theta(t)}(\mathbf{x}_i)$. This means we can also calculate the derivative of one of the network outputs $f_{\theta(t)}(\mathbf{x}_i)_j = f_{ij}$ to
\begin{align}
	\pAbl{}{t} f_{ij} &= \pAbl{f_{ij}}{\theta_k}\pAbl{\theta_k}{t}\nonumber\\
	&= \pAbl{f_{ij}}{\theta_k} \left(- \pAbl{\mathscr{L}}{f_{ij}} \pAbl{f_{ij}}{\theta_k} \right)\label{eq:NTKarisesFromHere}\\
	&= - \underbrace{\pAbl{f_{ij}}{\theta_k} \pAbl{f_{lm}}{\theta_k}}_\mathlarger{=\vcentcolon \Lambda_{iljm}}
	\pAbl{\mathscr{L}}{f_{lm}}.\nonumber
\end{align}
The rank 4 hypermatrix $\Lambda$ is what we call the Neural Tangent Kernel for SGD. We sorted the indices of this matrix so that the first two refer to the input points of $f$ and the last two refer to the components of the output dimensions of the neural network. Note that this NTK is derived directly from the update algorithm of the SGD for infinitely small $\eta$, the equations above don't hold for other optimization systems.\\
Another way to derive the NTK using an approximation of $\Delta \mathscr{L}$ for small $\eta$ can be seen around page 196 of \cite{ThePrinciplesOfDeepLearningTheory}. The NTK derived there also works for tensorial gradient descent with a learning rate tensor $\eta_{ij}$.
	\section{Interpretation}
	The NTK is defined as 
\begin{equation}\label{eq:NTKDefinition}
	\Lambda_{i,j,\alpha,\beta} = \nabla_\theta f_\theta(\mathbf{x}_i)_\alpha \cdot \nabla_\theta f_\theta(\mathbf{x}_j)_\beta.
\end{equation}
For now, let's assume that the neural network has only one output, which gets rid of the $\alpha$ and $\beta$ indices for us and makes the NTK a regular matrix. We can use this matrix to calculate the "time" derivative of the neural network output by 
\begin{equation}\label{eq:NTKExplanation}
	\pAbl{}{t}f_{\theta(t)}(\mathbf{x}_i) = \sum_j \Lambda_{i,j} \left(- \pAbl{\mathscr{L}}{f_{\theta(t)}(\mathbf{x}_j)}\right).
\end{equation}
this means that the evolution of the network output for input $\mathbf{x}_i$ is influenced by the outputs for other input values through the NTK. We can investigate this further by taking a look at the definition of the NTK above in \cref{eq:NTKDefinition}. \\

A mathematical kernel $K$ is defined as a function
\begin{equation}
	K(\mathbf{x}_i, \mathbf{x}_j) = \phi(\mathbf{x}_i)\cdot\phi(\mathbf{x}_i),
\end{equation}
with another so-called "feature map" function $\phi$ that maps the points $\mathbf{x}_i$ into a higher dimensional inner product space called the "feature space". The kernel assigns a scalar value to the two points by comparing their "features" via a scalar product in the feature space. In our case the feature map is $\nabla_\theta$ which maps our scalar network output onto a vector that contains the derivatives of the network output with respect to all of the various parameters. This is our feature space because what we're interested in is how moving in $\theta$ space changes the output values of our network. If we compare those two mappings we get a value that measures how closely the direction that increases $f_\theta(\mathbf{x}_i)$ in the most effective way (which is the direction the gradient points to) matches with the direction that increases $f_\theta(\mathbf{x}_j)$ most effectively. This also means that $\Lambda_{i,j}=0$ means that varying $\theta$ in the direction that changes $f_\theta(\mathbf{x}_i)$ most effectively results in 0 change for $f_\theta(\mathbf{x}_i)$.\\
Coming back to \cref{eq:NTKExplanation} the right side is the negative loss derivative with respect to $f_\theta(\mathbf{x}_j)$. Since we update the parameters in a way that minimizes the loss most effectively in gradient flow, the negative loss derivative with respect to $f_\theta(\mathbf{x}_j)$ describes how beneficial increasing $f_\theta(\mathbf{x}_j)$ is for our system. If we now multiply this with the NTK, which is a kernel that describes how similar $f_\theta(\mathbf{x}_i)$ and $f_\theta(\mathbf{x}_j)$ behave when changing theta, and sum everything up, we get the change of the function value $f_\theta(\mathbf{x}_i)$ as result.\\
In $\theta$ space, we can directly relate the evolution of $\theta$ to its negative loss derivative, because the parameters are what we are changing. For the derivative of the output values, we need the NTK as well, because we don't change the function value directly. If the loss derivative with respect to a output value tells the system that it should increase this output value, the system changes parameters in a way that increases this output value. The difference to the derivative of $\theta$ is that the change of parameters now doesn't reflect in just one output, but in every other output value as well. That's where the NTK arises in the equation. The NTK is in a way a translation of the SGD into the space of the outputs of the neural network.\\
All of the explanations above apply to the 4 dimensional hypermatrix form of the NTK for multidimensional neural network output as well, which can be seen when comparing \cref{eq:NTKExplanation} to \cref{eq:NTKarisesFromHere}.
	
	\chapter{Results}
	\section{Fisher Trace - NTK relation}
	For a neural network with $N$ parameters, the Fisher Information is a matrix with $N^2$ entries. To illustrate how large those resulting matrices are, let's consider a network containing 3 hidden layers with a respective width of 128 neurons that gets trained on the $28\times28$ matrices of the MNIST dataset. This network has over $130000$ parameters, which results in over $10^{10}$ entries in the FI. When saving the entries as 32-bit floating point numbers, the matrix would take over 67GB of space. Calculating  this matrix in under 1 hour would require over 2 million entries to be calculated per second. Calculating the full matrix to optimize training is therefore computationally inefficient. That's why we're going to look at an equation for calculating the trace of the Fisher Information, or Fisher Trace for short, through other mathematical objects.\\
Let's first inspect the Fisher Trace by using the chain rule of derivation as
\begin{equation}
	\begin{split}
		\mathrm{tr}(I) &= \sum_{\alpha = 1}^{N_P} I_{\alpha\alpha}\\
		&= \sum_{\alpha = 1}^{N_P} \left\{\underset{(\mathbf{x}_i, \mathbf{y}_i) \in D}{E} \left[\tAbl{}{\theta_\alpha}\ell(f_\theta(\mathbf{x}_i),\mathbf{y}_i)\cdot \tAbl{}{\theta_\alpha}\ell(f_\theta(\mathbf{x}_i),\mathbf{y}_i)\right]\right\}\\
		&= \sum_{\alpha = 1}^{N_P} \left\{ \frac{1}{N} \sum_{i = 1}^{N_D} \left[\sum_{a=1}^{N_O}\left(\pAbl{\ell}{f_\theta(\mathbf{x}_i)_a}\tAbl{f_\theta(\mathbf{x}_i)_a}{\theta_\alpha}\right)\cdot\sum_{b=1}^{N_O}\left(\pAbl{\ell}{f_\theta(\mathbf{x}_i)_b}\tAbl{f_\theta(\mathbf{x}_i)_b}{\theta_\alpha}\right)\right]\right\}\\
		&= \frac{1}{N} \sum_{i = 1}^{N_D} \sum_{a=1}^{N_O} \sum_{b=1}^{N_O} \Bigg[\pAbl{\ell}{f_\theta(\mathbf{x}_i)_a}\pAbl{\ell}{f_\theta(\mathbf{x}_i)_b} \cdot\underbrace{ \sum_{\alpha = 1}^{N_P} \left(\tAbl{f_\theta(\mathbf{x}_i)_a}{\theta_\alpha}\tAbl{f_\theta(\mathbf{x}_i)_b}{\theta_\alpha}\right)}_{\Lambda_{iiab}}\Bigg],
	\end{split} 
\end{equation}
with the amount of parameters $N_P$, the amount of dataset pairs $N_D$ and the output dimension of the network $N_O$. By identifying the entries of the NTK in the last line and simplifying the notation we can rewrite the relation as
\begin{equation}\label{eq:FisherNTKRelation}
	\mathrm{tr}(I) = \underset{(\mathbf{x}_i, \mathbf{y}_i) \in D}{E} \left[\sum_{a,b} \left(\pAbl{\ell}{f_\theta(\mathbf{x}_i)_a}\pAbl{\ell}{f_\theta(\mathbf{x}_i)_b} \cdot \Lambda_{iiab}\right)\right].
\end{equation}
Here it is important to note that there are two main parts on the right hand side: One is the loss derivative and the other is the NTK. Going further we will conduct some experiments to investigate which of those two terms is the dominant driving force in the evolution of the Fisher Trace. Specifically, we will look at the time evolution of the Fisher Trace in comparison the trace of the NTK to see how similar they evolve during training.

\subsection{Comparison of traces of NTK and Fisher Information}
To test how similar the evolution of the traces of NTK and Fisher Information are, we trained multiple networks on the MNIST dataset explained in \cref{sec:Datasets} and recorded the traces of the FI and NTK. We used networks consisting of 2 hidden layers that were equipped with the ReLU activation function. The network widths varied between $128$ and $784$. The losses used were $L_p$-Norm losses of order $n$ \cite{LpNormSource}
\begin{equation}
	d_n(f_\theta(\mathbf{x}),\mathbf{y}) = \sqrt[n]{\sum_{i=1}^{N_O} |f_\theta(\mathbf{x}_i) - \mathbf{y}_i|^n},
\end{equation}
Mean-power losses of order $n$
\begin{equation}
	d_n(f_\theta(\mathbf{x}),\mathbf{y}) = \frac{1}{N_O} \sum_{i=1}^{N_O} (f_\theta(\mathbf{x}_i)-\mathbf{y}_i)^n,
\end{equation}
and the softmax-ross-entropy loss \cite{LossExamplePaper}
\begin{equation}
	d(f_\theta(\mathbf{x}),\mathbf{y}) = \sum_{i=1}^{N_O} \mathbf{y}_i \log(\sigma_i(f_\theta(\mathbf{x}))),
\end{equation}
where we used the definition for distance $d$ from \cref{eq:DistanceLoss}, and a softmax function $\sigma_i$ from \cref{eq:softmax}.
\begin{figure}
	\centering
	\includegraphics{text/results/FisherNTKComparisonPlots/Triple_comparison_losses2_128.pdf}
	\caption{text}
\end{figure}
	
	
	
	\nocite{*}
	\printbibliography[title=Literature]
	\begin{appendices}
		\crefalias{chapter}{appendix}
		\chapter{Some mathematical proofs}
		\section{Proof of \cref{eq:FIforIndependentExperiments}}
\label{sec:ProofFIforIndependentExperiments}
For all variables $X^n$ that satisfy
\begin{equation}\label{eq:UsedInProofFIforIndependentExperiments}
	f(x^n|\theta) = \sum_{\alpha = 1}^n f(x_\alpha|\theta)
\end{equation} 
we can prove
\begin{equation}
	I_{X^n,ij} = \sum_\alpha I_{X_\alpha,ij}
\end{equation}
by
\begin{equation}
\begin{split}
	I_{X^n,ij} =& \underset{x^n \in X^n}{E} \left\{\tAbl{}{\theta_i}\log f(x^n|\theta) \tAbl{}{\theta_j}\log f(x^n|\theta)\right\}\\
	=& \sum_{x^n \in X^n} \left\{\left[\tAbl{}{\theta_i} f(x^n|\theta)\right] \left[\tAbl{}{\theta_j} f(x^n|\theta)\right]  \left[f(x^n|\theta)\right]^{-1} \right\}\\
	\stackrel{\ref{eq:UsedInProofFIforIndependentExperiments}}{=}& \sum_{x^n \in X^n} \left\{\tAbl{}{\theta_i} \left[\prod_\alpha f(x_\alpha|\theta)\right] \tAbl{}{\theta_j} \left[\prod_\beta f(x_\beta|\theta)\right] \left[ \prod_\gamma f(x_\gamma|\theta)\right]^{-1}\right\}\\
	=&\sum_{x^n \in X^n} \left\{\left(\prod_\alpha f(x_\alpha|\theta)\right)\left(\sum_\alpha \tAbl{}{\theta_i}\log f(x_\alpha|\theta)\right) \right. 
	\\
	&\qquad\quad\left.\left(\prod_\beta f(x_\beta|\theta)\right)\left(\sum_\beta \tAbl{}{\theta_i}\log f(x_\alpha|\theta)\right) \left[\prod_\gamma f(x_\gamma|\theta)\right]^{-1} \right\}\\
	=& \underset{x^n \in X^n}{E}\left\{ \left[\sum_\alpha\tAbl{}{\theta_i}\log f(x_\alpha|\theta)\right] \left[\sum_\beta\tAbl{}{\theta_i}\log f(x_\beta|\theta)\right]\right\}\\
	\stackrel{(*)}{=}& \underset{x^n \in X^n}{E} \left\{ \sum_\alpha \left[ \tAbl{}{\theta_i}\log f(x_\alpha|\theta)\right]\left[ \tAbl{}{\theta_i}\log f(x_\alpha|\theta)\right]\right\}\\
	=& \sum_\alpha \underset{x^n \in X^n}{E} \left\{\left[ \tAbl{}{\theta_i}\log f(x_\alpha|\theta)\right]\left[ \tAbl{}{\theta_i}\log f(x_\alpha|\theta)\right]\right\}\\
	=& \sum_\alpha I_{X_\alpha,ij}.
\end{split}
\end{equation}
The equality at $(*)$ holds because for all $\alpha \neq \beta$
\begin{equation}
	\begin{split}
		&\underset{x^n \in X^n}{E} \left\{ \left[\tAbl{}{\theta_i} \log f(x_\alpha|\theta)\right]\left[\tAbl{}{\theta_i} \log f(x_\beta|\theta)\right]\right\} \\
		\propto& \sum_{x_\alpha}\sum_{x_\beta} \left\{ \left[\tAbl{}{\theta_i}  f(x_\alpha|\theta)\right]\left[\tAbl{}{\theta_i}  f(x_\beta|\theta)\right]\right\}\\
		=& \sum_{x_\alpha} \left\{ \left[\tAbl{}{\theta_i}  f(x_\alpha|\theta)\right]\sum_{x_\beta}\left[\tAbl{}{\theta_i}  f(x_\beta|\theta)\right]\right\}\\
		=& \sum_{x_\alpha} \left\{ \left[\tAbl{}{\theta_i}  f(x_\alpha|\theta)\right]\sum_{x_\beta}\tAbl{}{\theta_i}\left[  f(x_\beta|\theta)\right]\right\}\\
		=& \sum_{x_\alpha} \left\{ \left[\tAbl{}{\theta_i}  f(x_\alpha|\theta)\right]\sum_{x_\beta}\tAbl{}{\theta_i}\left[ 1\right]\right\}\\
		=&\ 0.
	\end{split}
\end{equation}

		\chapter{MNIST experiment for trace comparison}
		\label{sec:TraceExperimentAppendix}
		This section contains more results of the experiment from \cref{sec:TraceComparisonExperiment}. The results for loss functions of order 4 and 6 for the network of width 128 are depicted in \cref{fig:MNISTTraceComparison1} and \cref{fig:MNISTTraceComparison2}. The results for loss functions of order 2, 4 and 6 for the network of width 784 are depicted in \cref{fig:MNISTTraceComparison3}, \cref{fig:MNISTTraceComparison4} and \cref{fig:MNISTTraceComparison5}. The Cross-entropy loss doesn't vary with order but is still depicted for reference in every plot.

\begin{figure}
	\centering
	\includegraphics[width=\textwidth]{text/results/FisherNTKComparisonPlots/Triple_comparison_losses4_128.pdf}
	\caption{Trace of NTK and Fisher information during 300 epochs of training on the MNIST dataset using the Adam optimizer. The network consisted of 2 hidden layers with a \emph{width of 128 neurons} that were equipped with the ReLU activation function. The loss functions used for optimization are denoted in each subplot. The solid line represents the mean value of 5 experiments, the translucent area represents one standard deviation from the mean value.}
	\label{fig:MNISTTraceComparison1}
\end{figure}
\begin{figure}
	\centering
	\includegraphics[width=\textwidth]{text/results/FisherNTKComparisonPlots/Triple_comparison_losses6_128.pdf}
	\caption{Trace of NTK and Fisher information during 300 epochs of training on the MNIST dataset using the Adam optimizer. The network consisted of 2 hidden layers with a \emph{width of 128 neurons} that were equipped with the ReLU activation function. The loss functions used for optimization are denoted in each subplot. The solid line represents the mean value of 5 experiments, the translucent area represents one standard deviation from the mean value.}
	\label{fig:MNISTTraceComparison2}
\end{figure}
\begin{figure}
	\centering
	\includegraphics[width=\textwidth]{text/results/FisherNTKComparisonPlots/Triple_comparison_losses2_784.pdf}
	\caption{Trace of NTK and Fisher information during 300 epochs of training on the MNIST dataset using the Adam optimizer. The network consisted of 2 hidden layers with a \emph{width of 784 neurons} that were equipped with the ReLU activation function. The loss functions used for optimization are denoted in each subplot. The solid line represents the mean value of 5 experiments, the translucent area represents one standard deviation from the mean value.}
	\label{fig:MNISTTraceComparison3}
\end{figure}
\begin{figure}
	\centering
	\includegraphics[width=\textwidth]{text/results/FisherNTKComparisonPlots/Triple_comparison_losses4_784.pdf}
	\caption{Trace of NTK and Fisher information during 300 epochs of training on the MNIST dataset using the Adam optimizer. The network consisted of 2 hidden layers with a \emph{width of 784 neurons} that were equipped with the ReLU activation function. The loss functions used for optimization are denoted in each subplot. The solid line represents the mean value of 5 experiments, the translucent area represents one standard deviation from the mean value.}
	\label{fig:MNISTTraceComparison4}
\end{figure}
\begin{figure}
	\centering
	\includegraphics[width=\textwidth]{text/results/FisherNTKComparisonPlots/Triple_comparison_losses6_784.pdf}
	\caption{Trace of NTK and Fisher information during 300 epochs of training on the MNIST dataset using the Adam optimizer. The network consisted of 2 hidden layers with a \emph{width of 784 neurons} that were equipped with the ReLU activation function. The loss functions used for optimization are denoted in each subplot. The solid line represents the mean value of 5 experiments, the translucent area represents one standard deviation from the mean value.}
	\label{fig:MNISTTraceComparison5}
\end{figure}
	\end{appendices}
	
\end{document}